%Template compiled by Alice lol
\documentclass[11pt]{article}
\usepackage{cs70}

%%%%%%%%%%%%%%%%%%%% name/id
\rfoot{\small Zehao Huang | 3033857597 | zehao@berkeley.edu}


%%%%%%%%%%%%%%%%%%%% Course/HW info
\newcommand*{\instr}{Babak Ayazifar and Satish Rao}
\newcommand*{\term}{Spring 2019}
\newcommand*{\coursenum}{CS 70}
\newcommand*{\coursename}{Discrete Mathematics and Probability Theory}
\newcommand*{\hwnum}{01}

%%%%%%%%%%%%%%%%%%%%%%%%%%%%%% Document Start %%%%%%%%%%%%%%%%%
\begin{document}

\question{Sundry}

Before you start your homework, state briefly how you worked on it. Who else did you work with? 
List names and email addresses. (In case of homework party, you can just describe the group.)

\begin{Answer}
	I have discussed the homework problems with the following people. I then wrote down the solutions
	myself.

	\begin{itemize}
		\item Jeffery Chen -- jefferyyc@berkeley.edu
	\end{itemize}
\end{Answer}

\newpage
\question{Miscellaneous Logic}

\begin{Parts}
	
	\Part Let the statement, $(\forall x \in \R)(\exists y \in \R) G(x, y)$, be true for 
	predicate $G(x, y)$.

	For each of the following statements, decide if the statement is certainly true, certainly 
	false, or possibly true, and justify your solution. (If possibly true, provide a specific 
	example where the statement is false and a specific example where the statement is true.)

	\begin{enumerate}[(i)]
		\item $G(3,4)$ 
		
		\begin{Answer}
			Possibly true. It is only guaranteed that when $x=3$, there is some 
			$y$ that will make $G(x,y)$ true. However, $y$ is not guaranteed to equal $4$. An
			example to make $G(3,4)$ false is let G be $x=y$. An example to make $G(3,4)$ true
			is let G be $x<y$. 
		\end{Answer}

		\item $(\forall x \in \R) G(x,3)$x
		
		\begin{Answer}
			Possibly true. It is only guaranteed that for x being any real number, there is 
			some $y$ that will make $G(x,y)$ true. However, $y$ is not guaranteed to equal $3$. 
			An example to make the statement true is let G be $y=3$. An example to make the 
			statement false is let G be $y \neq 3$.
		\end{Answer}

		\item $\exists y G(3,y)$
		
		\begin{Answer}
			Certainly true. It has been guaranteed for any value of x there exists a $y$ such 
			that $G(x,y)$ is true. As a result, when $x=3$, there must exist a $y \in \R$ such 
			that $G(x,y)$ is true.
		\end{Answer}

		\item $\forall y \neg G(3,y)$
		
		\begin{Answer}
			Certainly false. It has been guaranteed for any value of x there exists a $y$ such
			that $G(x,y)$ is true. As a result, when $x=3$, there must exist at least one 
			$y \in \R$ such that $G(x,y)$ is true. As a result, it cannot be true that all 
			$y \in \R$ will make $G(x,y)$ false. 
		\end{Answer}

		\item $\exists x G(x,4)$
		
		\begin{Answer}
			Possibly true. It is only guaranteed that for x being any real number, there is 
			some $y$ that will make $G(x,y)$ true. There might be an x where $y=4$ makes $G(x,y)$
			true, or there might be such an x. An example to make the statement true is let $G$ be
			$x=y$, where $x=4$ will satisfy the situation. An example to make the statement false
			is let $G$ be $y \neq 4$, in which case no $x \in \R$ can make $G(x,4)$ true. 
		\end{Answer}
	\end{enumerate}
	
	\newpage
	\Part Give an expression using terms involving $\lor$, $\land$ and ¬ which is true if and only 
	if exactly one of $X$, $Y$, and $Z$ is true.

	\begin{Answer}
		$(X \land \neg Y \land \neg Z) \lor (\neg X \land Y \land \neg Z) \lor (\neg X \land \neg Y \land Z)$
	\end{Answer}
	
\end{Parts}

\newpage
\question{Propositional Practice}

Convert the following English sentences into propositional logic and the following 
propositions into English. State whether or not each statement is true with brief justification. 

\begin{Parts}
	
	\Part There is a real number which is not rational.

	\begin{Answer}
		$(\exists x \in \R)(x \notin \Q)$ \\ 
		This is true, an example would be $\sqrt{2}$. 
	\end{Answer}

	\Part All integers are natural numbers or are negative, but not both.

	\begin{Answer}
		$(\forall x \in \Z)((\neg (x \in \N) \land (x < 0)) \lor ((x \in \N) \land \neg (x < 0)))$ \\
		This is true, since if an integer is negative, it is less than $0$, meaning that it is not
		a natural number. If the integer is not negative, it is either exactly $0$ or it is positive, 
		meaning that it is a natural number. 
	\end{Answer}

	\Part If a natural number is divisible by 6, it is divisible by 2 or it is divisible by 3.

	\begin{Answer}
		$(\forall x \in \N)((6 \mid x) \implies ((2 \mid x) \lor (3 \mid x)))$ \\ 
		This is true, since if $6 \mid x$, $\exists q \in \Z$ such that $x=6q$. It is obvious that 
		$x=2(3q)$ or $x=3(2q)$, and $3q,2q \in \Z$. Thus, $2 \mid x$ or $3 \mid x$.
	\end{Answer}
	
	\Part $(\forall x \in \R)(x \in \C)$

	\begin{Answer}
		All real numbers are complex numbers. \\
		It is true, since $\R \subset \C$. 
	\end{Answer}

	\Part $(\forall x \in \Z)(((2|x) \lor (3|x)) \implies (6|x))$

	\begin{Answer}
		If an integer is divisible by 2 or divisible by 3, it is divisible by 6. \\ 
		It is false. A counterexample would be $x=2$. 
	\end{Answer}

	\Part $(\forall x \in \N)((x>7) \implies ((\exists a,b \in \N)(a+b=x)))$

	\begin{Answer}
		All natural numbers larger than 7 can be represented as the sum of two natural numbers. \\
		It is obviously true, since it can alway be represented as the sum of $1$ and $n-1$. 
	\end{Answer}
	
\end{Parts}

\newpage
\question{Prove or Disprove}

\begin{Parts}
	
	\Part $(\forall n \in \N)$ if $n$ is odd then $n^2+2n$ is odd.

	\begin{Answer}
		True. $n$ is odd $\implies (\exists k \in \N)(n = 2k + 1)$. $n^2 + 2n = (2k + 1)^2 + 2(2k + 1)
		= 2(2k^2 + 4k + 1) + 1$. Let $l = 2k^2 + 4k + 1$. It's obvious that $l \in \N$ and $n^2 + 2n = 
		2l + 1$. As a result, we conclude that $n^2 + 2n$ is odd. 
	\end{Answer}

	\Part $(\forall x,y \in \R) min(x,y) = (x + y - |x - y|) / 2$

	\begin{Answer}
		True. We will prove by cases. Either $x \leq y$ or $x > y$. If $x \leq y$, $min(x,y)=x$, $(x + y
		 - |x - y|) / 2 = ((x+y)-(y-x))/2 = x$. If $x > y$, $min(x,y)=y$, $(x+y-|x-y|)/2 = ((x+y)-(x-y))/2
		  = y$. As a result, we conclude that $(\forall x,y \in \R) min(x,y) = (x + y - |x - y|) / 2$. 
	\end{Answer}

	\Part $(\forall a,b \in \R)$ if $a+b \leq 10$ then $a \leq 7$ or $b \leq 3$.

	\begin{Answer}
		True. We prove by contraposition. The original proposition can be rewritten as $(\forall a,b \in \R)
		(a+b \leq 10) \implies ((a \leq 7) \lor (b \leq 3))$. Negating the conclusion gives us $((a > 7) \land
		(b > 3)) \implies (a + b > 10)$, which directly contradicts the hypothesis. As a result, we conclude 
		that $(\forall a,b \in \R)$ if $a+b \leq 10$ then $a \leq 7$ or $b \leq 3$.
	\end{Answer}

	\Part $(\forall r \in \R)$ if $r$ is irrational then $r + 1$ is irrational.

	\begin{Answer}
		True. We prove by contraposition. The original proposition can be rewritten as $(\forall r \in \R)
		(r \notin \Q) \implies (r + 1 \notin \Q)$. We assume that $r + 1 \in \Q$, then $(\exists a,b \in \Z)
		r+1=\frac{a}{b}$. Then $r=\frac{a-b}{b}$, and since $a-b \in \Z$ and $b \in \Z$, $r \in \Q$. This 
		contradicts the hypothesis. Thus, we can conclude here. 
	\end{Answer}

	\Part $(\forall n \in \Z_+) 10n^2 > n!$.

	\begin{Answer}
		False. We give cases that will contradict the proposition. Let $n = 10$, $10n^2 = 10^3 = 1000$, 
		$10! = 3,628,800$, $10n^2 < n!$, the proposition is not true. 
	\end{Answer}

\end{Parts}

\newpage
\question{Preserving Set Operations}

For a function $f$, define the image of a set $X$ to be the set $f(X) = \{y~|~y = f(x) \text{ for some } 
x \in X\}$. Define the inverse image or preimage of a set $Y$ to be the set $f^{-1}(Y) = \{x~|~f(x) \in Y\}$. 
Prove the following statements, in which $A$ and $B$ are sets. By doing so, you will show that inverse 
images preserve set operations, but images typically do not.

\begin{Parts}
	
	\Part $f^{-1}(A \cup B) = f^{-1}(A) \cup f^{-1}(B)$.

	\begin{Answer}
		First, $\forall x \in f^{-1}(A \cup B)$, $f(x) \in (A \cup B) \implies (f(x) \in A) \lor (f(x) \in B) 
		\implies (x \in f^{-1}(A)) \lor (x \in f^{-1}(B)) \implies x \in (f^{-1}(A) \cup f^{-1}(B)) \implies 
		f^{-1}(A \cup B) \subseteq (f^{-1}(A) \cup f^{-1}(B))$. Then, $\forall x \in f^{-1}(A) \cup f^{-1}(B)$, 
		$(x \in f^{-1}(A)) \lor (x \in f^{-1}(B)) \implies (f(x) \in A) \lor (f(x) \in B) \implies f(x) \in 
		(A \cup B) \implies (f^{-1}(A) \cup f^{-1}(B)) \subseteq f^{-1}(A \cup B)$. We conclude. 
	\end{Answer}
	
	\Part $f^{-1}(A \cap B) = f^{-1}(A) \cap f^{-1}(B)$.

	\begin{Answer}
		First, $\forall x \in f^{-1}(A \cap B)$, $f(x) \in (A \cap B) \implies (f(x) \in A) \land (f(x) \in B) 
		\implies (x \in f^{-1}(A)) \land (x \in f^{-1}(B)) \implies x \in f^{-1}(A) \cap f^{-1}(B) \implies 
		f^{-1}(A \cap B) \subseteq f^{-1}(A) \cap f^{-1}(B)$. Then, $\forall x \in f^{-1}(A) \cap f^{-1}(B)$,
		$(x \in f^{-1}(A)) \land (x \in f^{-1}(B)) \implies (f(x) \in A) \land (f(x) \in B) \implies f(x) \in 
		A \cap B \implies x \in f^{-1}(A \cap B) \implies f^{-1}(A) \cap f^{-1}(B) \subseteq f^{-1}(A \cap B)$.
		We conclude. 
	\end{Answer}

	\Part $f^{-1}(A \setminus B) = f^{-1}(A) \setminus f^{-1}(B)$.
	
	\begin{Answer}
		First, $\forall x \in f^{-1}(A \setminus B)$, $f(x) \in (A \setminus B) \implies (f(x) \in A) \land 
		(f(x) \notin B) \implies (x \in f^{-1}(A)) \land (x \notin f^{-1}(B)) \implies x \in f^{-1}(A) \setminus 
		f^{-1}(B) \implies f^{-1}(A \setminus B) = f^{-1}(A) \setminus f^{-1}(B)$. Then, $\forall x \in f^{-1}(A)
		\setminus f^{-1}(B)$, $(x \in f^{-1}(A)) \land (x \notin f^{-1}(B)) \implies (f(x) \in A) \land (f(x) 
		\notin B) \implies f(x) \in (A \setminus B) \implies x \in f^{-1}(A \setminus B) \implies f^{-1}(A) 
		\setminus f^{-1}(B) \subseteq f^{-1}(A \setminus B)$. We conclude. 
	\end{Answer}
	
	\Part $f(A \cup B) = f(A) \cup f(B)$.
	
	\begin{Answer}
		First, $\forall x \in A \cup B$, $(x \in A) \lor (x \in B)$. $x \in A \implies f(x) \in f(A)$, $x \in B
		\implies f(x) \in f(B)$, $(x \in A) \lor (x \in B) \implies f(x) \in f(A) \cup f(B) \implies f(A \cup B) 
		\subseteq f(A) \cup f(B)$. Then, $\forall f(x) \in f(A) \cup f(B)$, $(f(x) \in f(A)) \lor (f(x) \in f(B))
		\implies (x \in A) \lor (x \in B) \implies x \in (A \cup B) \implies f(x) \in f(A \cup B)$, $f(A) \cup f(B)
		\subseteq f(A \cup B)$. We conclude. 
	\end{Answer}

	\Part $f(A \cap B) \subseteq f(A) \cap f(B)$, and give an example where equality does not hold.
	
	\begin{Answer}
		$\forall x \in A \cap B$, $f(x) \in f(A \cap B) \implies (f(x) \in A) \land (f(x) \in B) \implies (x \in A)
		\land (x \in B) \implies f(x) \in f(A) \cap f(B)$. An example where equality does not hold is $A = \{2, 0, -1\}$, 
		$B = \{-2, 0, 1\}$, then $A \cap B = \{0\}$. Let $f(x) = x^2$, $f(A \cap B) = \{0\}$, $f(A) = \{0, 1, 4\}$, 
		$f(B) = \{0, 1, 4\}$, $f(A) \cap f(B) = \{0, 1, 4\}$. The equality does not hold.  
	\end{Answer}

	\Part $f(A \setminus B) \supseteq f(A) \setminus f(B)$, and give an example where equality does not hold.
	
	\begin{Answer}
		$\forall x \in A \setminus B$, $f(x) \in f(A \setminus B) \implies (f(x) \in f(A)) \land (f(x) \notin f(B))
		\implies (x \in A) \land (x \notin B) \implies f(x) \in f(A) \setminus f(B)$. An example where equality does 
		not hold is $A = \{1, -1, 0\}$, $B = \{1\}$, $f(x) = x^2$. Then $f(A \setminus B) = \{0, 1\}$, $f(A) = \{0, 1\}$,
		$f(B) = \{1\}$, $f(A) \setminus f(B) = \{0\}$. The equality does not hold. 
	\end{Answer}
	
\end{Parts}

\newpage
\question{Hit or Miss?}

State which of the proofs below is correct or incorrect. For the incorrect ones, please explain clearly where the 
logical error in the proof lies. Simply saying that the claim or the induction hypothesis is false is \emph{not} 
a valid explanation of what is wrong with the proof. You do not need to elaborate if you think the proof is correct.

\begin{Parts}
	\Part
		\textbf{Claim:} For all positive numbers $n \in \mathbb{R}$, $n^2 \ge n$.
		\begin{proof}
			The proof will be by induction on $n$.	\\
			\emph{Base Case:} $1^2 \ge 1$. It is true for $n=1$.	\\
			\emph{Inductive Hypothesis:} Assume that $n^2 \ge n$. 	\\
			\emph{Inductive Step:} We must prove that $(n+1)^2 \ge n+1$.
			Starting from the left hand side,
			\begin{align*}
				(n+1)^2 &= n^2+2n+1 \\
						&\ge n+1.
			\end{align*}
			Therefore, the statement is true.
		\end{proof}
		
	\begin{Answer}
		The proof is incorrect. The proof only proves that the statement is true for all positive integers. However, the domain 
		of n here is all real numbers, and the proof above fails to address any decimal number. A counterexample to the claim would 
		be $n=0.1$, $0.1^2=0.01<0.1$, the claim does not hold. 
	\end{Answer}

	\Part
		\textbf{Claim:} For all negative integers $n$, $(-1) + (-3) + \cdots + (2n+1) = -n^2$.
		\begin{proof}
			The proof will be by induction on $n$.	\\
			\emph{Base Case:} $-1 = -(-1)^2$. It is true for $n=-1$. \\
			\emph{Inductive Hypothesis:} Assume that $(-1) + (-3) + \cdots + (2n+1) = -n^2$. \\
			\emph{Inductive Step:} We need to prove that the statement is also true for $n-1$ if it is true for $n$, that is,
			 $(-1) + (-3) + \cdots + (2(n-1)+1) = -(n-1)^2$. Starting from the left hand side,
			\begin{align*}
				(-1) + (-3) + \cdots + (2(n-1)+1)
				&= ((-1) + (-3) + \cdots + (2n+1))+(2(n-1)+1)	\\
				&= -n^2 + (2(n-1)+1)				\quad \text{(Inductive Hypothesis)}\\
				&= -n^2 + 2n - 1					\\
				&= -(n^2 - 2n + 1) \\
				&= -(n-1)^2.
			\end{align*}
			Therefore, the statement is true.
		\end{proof}

	\begin{Answer}
		% The proof is incorrect. The inductive hypothesis should assume that for an arbitrary $n=k$, $(-1)+(-3)+\ldots+(2k+1)
		% =-k^2$. Then in the inductive step, it should use this arbitrary $k$ to do all relevant work. 
		The proof is correct. 
	\end{Answer}

	\Part
		\textbf{Claim:} For all nonnegative integers $n$, $2n=0$.
		\begin{proof}
			We will prove by strong induction on $n$.	\\
			\emph{Base Case:} $2 \times 0 = 0$. It is true for $n=0$.	\\
			\emph{Inductive Hypothesis:} Assume that $2k=0$ for all $0 \le k \le n$.	\\
			\emph{Inductive Step:} We must show that $2(n+1)=0$. Write $n+1 = a+b$ where $0 < a,b \le n$.
			From the inductive hypothesis, we know $2a = 0$ and $2b=0$, therefore,
			\begin{align*}
				2(n+1) = 2(a+b) = 2a + 2b = 0+0 =0.
			\end{align*}
			The statement is true.
		\end{proof}

	\begin{Answer}
		The proof is incorrect. In the first inductive step, when $n=0$, $0<a,b\leq 0$, $n+1$ cannot be rewritten as 
		the sum of $a$ and $b$. Thus the inductive step does not hold, and the statement is proven to be false. 
	\end{Answer}
\end{Parts}

\newpage
\question{Badminton Ranking}

A team of $n$ ($n \geq 2$) badminton players held a tournament, where every person plays with every other person exactly 
once, and there are no ties. Prove by induction that after the tournament, we can arrange the $n$ players in a sequence, 
so that every player in the sequence has won against the person immediately to the right of him.

\begin{Answer}
	\begin{proof}
		The proof will be by induction on $n$.	\\
		\emph{Base Case:} There must be a winner and a loser in 2 players. The claim is true for $n=2$. \\
		\emph{Inductive Hypothesis:} Assume that for $k$ players, there is a sequence in which every player has won against 
		the person immediately to the right of him. \\ 
		\emph{Inductive Step:} We need to prove that the statement is also true for $k+1$ if it is true for $k$. We take out
		the correct sequence for the rest $k$ players and consider player $k+1$. If he wins all the contests, we put him to 
		the front of the sequence. If he loses all his contests, we put him to the right most end of the sequence. If neither
		of the cases apply, we just search the entire sequence and insert the player between two other players, one of whom 
		he wins and the other he loses. We start from the left end and move on one player per step. There must exist a case 
		where the new player is won by the player to his left and wins the player to his right. Otherwise it indicates that 
		the player defeats or loses all the contests, which have been discussed above. \\
		Therefore, the statement is true.
	\end{proof}
\end{Answer}

\end{document}
