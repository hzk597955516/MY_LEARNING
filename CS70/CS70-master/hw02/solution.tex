%Template compiled by Alice lol
\documentclass[11pt]{article}
\usepackage{cs70}

%%%%%%%%%%%%%%%%%%%% name/id
\rfoot{\small Zehao Huang | 3033857597 | zehao@berkeley.edu}


%%%%%%%%%%%%%%%%%%%% Course/HW info
\newcommand*{\instr}{Babak Ayazifar and Satish Rao}
\newcommand*{\term}{Spring 2019}
\newcommand*{\coursenum}{CS 70}
\newcommand*{\coursename}{Discrete Mathematics and Probability Theory}
\newcommand*{\hwnum}{02}

%%%%%%%%%%%%%%%%%%%%%%%%%%%%%% Document Start %%%%%%%%%%%%%%%%%
\begin{document}
    \question{Sundry}

    \begin{Answer}
        I typeset the homework using \LaTeX and discussed the homework with the following people. 
        \begin{itemize}
            \item Yijia Chen, yijia.chen@berkeley.edu
        \end{itemize}
    \end{Answer}

    \newpage
    \question{Induction on Reals}

    \begin{Answer}
        We use $p_i,q_i \in \R$ to denote the height of Bug Bob after $i$ jumps assuming its initial height is $p$ or $q$. 
        We first prove the helper claim.\\
        \textbf{Claim:} $\forall p_0,q_0 \in \R, n \in \N_+, p_0 < q_0 \implies p_n < q_n$.
		\begin{proof}
			The proof will be by induction on $n$.  \\
			\emph{Base Case:} $n = 1$, $p_1 = (p_0 + 1) / 2 < (q_0 + 1) / 2 = q_1$. The claim holds.  \\
			\emph{Inductive Hypothesis:} Assume that $p_k < q_k$. \\
            \emph{Inductive Step:} We must prove that $p_{k+1} < q_{k+1}$.
            \begin{align*}
				p_{k+1}=\frac{p_k+1}{2}=\frac{p_k}{2}+\frac{1}{2}<\frac{q_k}{2}+\frac{1}{2}=\frac{q_k+1}{2}=q_{k+1}
            \end{align*}
            Therefore, the claim is true.
        \end{proof}
        Thus, $\forall h_0 \in \R$, as long as we can prove that Bob will fall into Sally's net in a finite
        number of seconds with the initial height of $\ceil{h_0}$, we can prove that Bob will fall into Sally's
        net in a finite number of seconds. In short, $P(\ceil{h_0}) \implies P(h_0)$. \\ 
        \textbf{Claim:} $\forall h_0 \in \R$, $h_i=(h_{i-1}+1)/2$, $(\exists q \in \N)(h_q \leq 2)$.
        \begin{proof}
            The proof will be by induction on $\ceil{h_0}$. Note that $\ceil{h_0} \in \N$. \\
            \emph{Base Case:} $\ceil{h_0} = 3 \iff 2<h_0\leq 3 \iff 1.5<h_1\leq 2$. The claim holds. \\
            \emph{Inductive Hypothesis:} Assume the claim holds for all $h$ such that $3 \leq \ceil{h} < \ceil{h_0}$. \\
            \emph{Inductive Step:} We must show that the claim holds for $\ceil{h_0}$. 
            \begin{align*}
                h_0-h_1=h0-(h_0+1)/2=\frac{h_0-1}{2}>1\iff \ceil{h_0}>3 \iff h_0>3
            \end{align*}
            Thus $\ceil{h_1}<\ceil{h_0}$. Thus, using one more seconds than the case of $h_1$, Bob will also fall into
            Sally's net in the case of $h_0$. Combining the claim we proved above, we conclude. 
        \end{proof}
    \end{Answer}

    \newpage
    \question{Grid Induction}

    \begin{Answer}
        We can observe that no matter however the the Pacman goes, the sum of his coordinates always reduces by one. 
        We are thus going to prove the strengthened claim below. \\
        \textbf{Claim:} For a Pacman on $(i,j)$, after $i+j$ walks he will reach $(0,0)$.
        \begin{proof}
            Let $n=i+j$. This proof will be by induction on $n$. \\
            \emph{Base Case:} $n=1$, the Pacman is at $(0,1)$ or $(1,0)$. After $1$ step, the Pacman will be at $(0,0)$. \\
            \emph{Inductive Hypothesis:} Assume when $n=k(k>1)$, after $k$ steps, the Pacman will be at $(0,0)$. \\
            \emph{Inductive Step:} We must show that when $n=k+1$, after $k+1$ steps, the Pacman will be at $(0,0)$. 
            As $k+1=i+j$, after $1$ step, the Pacman will be at either $(i-1,j)$ or $(i,j-1)$. In either cases, $i+j-1=k$.
            According to the inductive hypothesis, we know that after $k$ steps the Pacman will be at $(0,0)$. \\ 
            We conclude that the claim is true. 
        \end{proof}
    \end{Answer}

    \newpage
    \question{Stable Marriage}

    \begin{Parts}
        \Part Run on this instance. Show each day of the algorithm, and give the resulting matching. 

        \begin{Answer}
            \begin{itemize}
                \item Day 1. $A,B,C$ proposes to $1$, $D$ proposes to $3$. $1$ rejects $B,C$ and puts $A$ on the string. $3$ puts $D$ on the string. 
                \item Day 2. $B,C$ proposes to $3$. $3$ rejects $C,D$ and puts $B$ on the string. 
                \item Day 3. $C$ proposes to $2$, $D$ proposes to $1$. $2$ puts $C$ on the string, $1$ rejects $A$ and accepts $D$. 
                \item Day 4. $A$ proposes to $2$. $2$ rejects $C$ and accepts $A$. 
                \item Day 5. $C$ proposes to $4$. $4$ puts C on the string. $3$ accepts $B$. $4$ accepts $C$. 
            \end{itemize}
            The final pairing would be $\{(D,1),(A,2),(B,3),(C,4)\}$.
        \end{Answer}

        \Part Prove that the modification will not change what pairing the algorithm outputs. 

        \begin{Answer}
            \begin{proof}
                Suppose that the modification will change the pairing the algorithm outputs. Suppose that $\{b^*, g^*\}$ and $\{b, g\}$ 
                are paired in the output of the algorithm before the modification; in the new pairing output, boy $b$ and girl $g^*$ got 
                paired together. We'll continue the proof by cases.
                \begin{itemize}
                    \item $b$ and $g^*$ end up together because $g^*$ rejects $b^*$. \\
                          Assume $g^*$ is the first girl running into this situation. We know that $g^*$ prefers $b$ to $b^*$. However, in
                          the original pairing $g^*$ didn't end up together with $b$ because $b$ never proposed to $g^*$ before he proposed 
                          to his partner $g$. Thus in the new pairing, $b$ proposes to $g^*$ because he was rejected by his original partner
                          $g$. This is a contradiction of the well ordering principle. 
                    \item $b$ and $g^*$ end up together because $g^*$ didn't receive the proposal from $b^*$. \\
                          Assume $g^*$ is the first girl running into this situation. We know that $g^*$ prefers $b^*$ to $b$. However, in 
                          the current pairing $b^*$ didn't propose to $g^*$ because he was accepted by another girl. The fact that this girl
                          rejected him in the original pairing indicates that her original date didn't propose to her in the new pairing. This
                          is a contradiction of the well ordering principle.  
                \end{itemize}
                As a result, the order of proposal does not affect the final pairing. Thus, the relaxation of boys' rules will not affect 
                the final output of the algorithm.
            \end{proof}
        \end{Answer}

    \end{Parts}

    \newpage
    \question{The Better Stable Matching}

    \begin{Parts}

        \Part $R=\{(A,4),(B,3),(C,1),(D,2)\}$, $R'=\{(A,3),(B,4),(C,2),(D,1)\}$

        \begin{Answer}
            $R \land R' = \{(A,3),(B,4),(C,1),(D,2)\}$. $R \land R'$ is also stable because there is no rogue couple in the pairing. 
        \end{Answer}

        \Part Prove that for any matchings $R,R'$, no man prefers $R$ or $R'$ to $R \land R'$. 

        \begin{Answer}
            \begin{proof}
                We will proceed the proof by contradiction. \\
                We assume that there exists a man that prefers $R$ or $R'$ to $R \land R'$. \\
                We assume that the man $m$'s date in $R$ is $g$, and his date in $R'$ is $g'$. 
                \begin{itemize}
                    \item If $m$ prefers $R$, then $m$'s date in $R \land R'$ must be $g'$ and $m$ must prefer $g$ to $g'$. According to 
                    the definition of $R \land R'$, however, $m$ must prefer $g'$ to $g$. There is a contradiction. 
                    \item If $m$ prefers $R'$, then $m$'s date in $R \land R'$ must be $g$ and $m$ must prefer $g'$ to $g$. According to 
                    the definition of $R \land R'$, however, $m$ must prefer $g$ to $g'$. There is a contradiction. 
                \end{itemize}
                Thus we come to conclude that for any matchings $R,R'$, no man prefers $R$ or $R'$ to $R \land R'$. 
            \end{proof}
        \end{Answer}

        \Part $m$ prefers $R$ to $R'$ and $w$ prefers $R'$ to $R$; or $m$ prefers $R'$ to $R$ and $w$ prefers $R$ to $R'$.
        
        \begin{Answer}
            \begin{proof}
                Let $M$ and $W$ denote the sets of mens and women respectively that prefer $R$ to $R'$, and $M'$ and $W'$ the sets of 
                men and women that prefer $R'$ to $R$.

                First we note that $|M|+|M'|=|W|+|W'|$. For each man $m$, he either prefers $R$ or $R'$. Let $n$ be the total number 
                of men or women. $|M|+|M'|=n$ and $|W|+|W'|=n$. Thus $|M|+|M'|=|W|+|W'|$.

                We then observe that $|M|\leq|W'|$. For every man $m$ in $M$, $g$ is $m$'s date in $R$ and $g^*$ is $m$'s date in $R'$. 
                $m \in M \implies$ $m$ prefers $g$ to $g^*$. Let $m^*$ be $g^*$'s date in $R$. If $g^*$ prefers $m^*$, $(m,g^*)\in R'$ 
                will become a rogue couple because they both prefer someone else. This contradicts the fact that $R'$ is stable. Thus 
                $m \in M \implies g^* \in W'$. As a result, $|M|\leq|W'|$. The same logic proves that $|M'|\leq|W|$.

                Considering $|M|+|M'|=|W|+|W'|$ we see that $|M'|=|W|$ and $|M|=|W'|$. Thus for all men in $M$, all their dates are in 
                $W'$, and for all men not in $M$, they must be in $M'$, and their dates must be in $|W|$. The claim is proved. 
            \end{proof}
        \end{Answer}

        \Part for any stable $R$ and $R'$, $R \land R'$ is also a matching and it is stable.

        \begin{Answer}
            \textbf{Claim:} $R \land R'$ is a matching. 
            \begin{proof}
                We will proceed by contradiction. \\ 
                \emph{Negation of Conclusion:} We assume that $R \land R'$ is not a matching. \\
                \emph{Proof by cases:} If $R \land R'$ is not a matching, one of the below cases must hold.
                \begin{itemize}
                    \item Two boys $b_1,b_2$ both end up with $g_1$.
                        \begin{itemize}
                            \item $(b_1,g_1)\in R \land (b_2,g_1)\in R'$. Thus $b_1$ prefers $R$ and $b_2$ prefers $R'$. However, 
                            according to part (c), $g_1$ would prefer $R$ and $R'$. We reach a contradiction. 
                            \item $(b_2,g_1)\in R \land (b_1,g_1)\in R'$. Thus $b_2$ prefers $R$ and $b_1$ prefers $R'$. However, 
                            according to part (c), $g_1$ would prefer $R$ and $R'$. We reach a contradiction. 
                        \end{itemize} 
                    \item Two girls $g_1,g_2$ both end up with $b_1$.
                        \begin{itemize}
                            \item $(b_1,g_1)\in R \land (b_1,g_2)\in R'$. Thus $g_1$ prefers $R$ and $g_2$ prefers $R'$. However, 
                            according to part (c), $b_1$ would prefer $R$ and $R'$. We reach a contradiction. 
                            \item $(b_1,g_2)\in R \land (b_1,g_1)\in R'$. Thus $g_2$ prefers $R$ and $g_1$ prefers $R'$. However, 
                            according to part (c), $b_1$ would prefer $R$ and $R'$. We reach a contradiction. 
                        \end{itemize} 
                \end{itemize}
                Thus, $R \land R'$ must be a matching. 
            \end{proof}

            \textbf{Claim:} $R \land R'$ is stable. 
            \begin{proof}
                We will proceed by contradiction. \\
                \emph{Negation of Conclusion:} We assume that $R \land R'$ is not stable. \\
                There exists two pairs $(b,g),(b^*,g^*)$ in $R \land R'$ where $b$ prefers $g^*$ to $g$ and $g^*$ prefers $b$ to $b^*$. 
                If $b$ prefers $R$, then in $R'$, $b$'s date is either $g$ or a $g'$ $b$ likes less. Since $R$ and $R'$ are stable and 
                $(b,g) \in R$, then $(b^*,g^*) \in R'$. In $R'$, $b$ and $g^*$ still form a rouge couple because $g^*$ likes $b$ better 
                than $b^*$ and since $b$'s current date is at most as good as $g$, $b$ still prefers $g^*$. This contradicts the fact 
                that $R'$ is stable. The same contradiction can be derived if we assume $b$ prefers $R'$. \\
                We can conclude that $R \land R'$ is stable. 
            \end{proof}
        \end{Answer}

    \end{Parts}

    \newpage
    \question{Examples or It's Impossible}

    \begin{Parts}
        
        \Part Every man get his first choice. 

        \begin{Answer}
            \begin{center}
                \begin{tabular}{|c|c||c|c|}\hline
                    men&preferences&women & preferences \\
                    \hline
                    A& 1$>$2$>$3 &1& A$>$B$>$C \\
                    \hline
                    B& 2$>$3$>$1 &2& B$>$C$>$A \\
                    \hline
                    C& 3$>$1$>$2 &3& C$>$A$>$B \\
                    \hline
                \end{tabular}
            \end{center}
        \end{Answer}

        \Part Every women gets her first choice, even though her first choice does not prefer her the most.
        
        \begin{Answer}
            \begin{center}
                \begin{tabular}{|c|c||c|c|}\hline
                    men&preferences&women & preferences \\
                    \hline
                    A& 3$>$1$>$2 &1& A$>$B$>$C \\
                    \hline
                    B& 1$>$2$>$3 &2& B$>$C$>$A \\
                    \hline
                    C& 1$>$3$>$2 &3& C$>$A$>$B \\
                    \hline
                \end{tabular}
            \end{center}
        \end{Answer}

        \Part Every woman gets her last choice. 

        \begin{Answer}
            \begin{center}
                \begin{tabular}{|c|c||c|c|}\hline
                    men&preferences&women & preferences \\
                    \hline
                    A& 2$>$3$>$1 &1& A$>$B$>$C \\
                    \hline
                    B& 3$>$1$>$2 &2& B$>$C$>$A \\
                    \hline
                    C& 1$>$2$>$3 &3& C$>$A$>$B \\
                    \hline
                \end{tabular}
            \end{center}
        \end{Answer}

        \Part Every man gets his last choice.

        \begin{Answer}
            Not possible. \\
            \textbf{Claim:} It's impossible that every man gets his last choice. 
            \begin{proof}
                We will proceed by contradiction. \\
                Assume every man gets his last choice. Then on the last day every man proposes to his least favorable woman. 
                As a result, before the last day, every man has been rejected by $n-1$ women, and every woman has rejected $n-1$ 
                man. However, the fact that every women has rejected at least one man indicates that there has been a day when
                every woman has been paired with some man and the algorithm should have terminated earlier. Thus we arrive at a 
                contradiction and the claim holds.  
            \end{proof}
        \end{Answer}

        \Part A man who is second on every women's list gets his last choice. 

        \begin{Answer}
            Possible. Example given below.
            \begin{center}
                \begin{tabular}{|c|c||c|c|}\hline
                    men&preferences&women & preferences \\
                    \hline
                    A& 1$>$3$>$2 &1& A$>$B$>$C \\
                    \hline
                    B& 2$>$1$>$3 &2& C$>$B$>$A \\
                    \hline
                    C& 2$>$1$>$3 &3& A$>$B$>$C \\
                    \hline
                \end{tabular}
            \end{center}
        \end{Answer}

    \end{Parts}

    \newpage
    \question{Short Answer: Graphs}

    \begin{Parts}

        \Part connected components?

        \begin{Answer}
            3
        \end{Answer}

        \Part edges removed? 

        \begin{Answer}
            7
        \end{Answer}
        
        \Part True or False? 

        \begin{Answer}
            False. \\
            An n-dimensional hyper-cube has $f(n)=2^{n-1}+2f(n-1)$ edges, while $K_n$ has $g(n)=n(n-1)/2$ edges. 
            We will prove the following claim. \\
            \textbf{Claim:} $n\geq3 \implies f(n) > g(n)$.
            \begin{proof}
                The proof will be by induction on $n$. \\
                \emph{Base case:} $n=3$, $f(3)=12$, $g(3)=3$. \\
                \emph{Inductive hypothesis:} Assume $n=k$, $f(k)>g(k)$. \\
                \emph{Inductive step:} We must show that $f(k+1)>g(k+1)$. 
                \begin{align*}
                    f(k+1)=2^k+2f(k)>2^k+2g(k)=2^k+k(k-1)=(2^k-2k)+2g(k+1)>g(k+1)
                \end{align*}
                Thus we know that an n-dimensional hyper-cube has more edges than a $K_n$. 
            \end{proof}
        \end{Answer}

        \Part number of cycles to cover a complete graph. 

        \begin{Answer}
            $\frac{n-1}{2}$
        \end{Answer}

        \Part Give a set of edge-disjoint Hamiltonian cycles that covers the edges of K5.
        
        \begin{Answer}
            \begin{itemize}
                \item Cycle 1: $(0,1),(1,2),(2,3),(3,4),(4,0)$
                \item Cycle 2: $(0,2),(2,4),(4,1),(1,3),(3,0)$
            \end{itemize}
        \end{Answer}
    \end{Parts}
\end{document}