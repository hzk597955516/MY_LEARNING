%Template compiled by Alice lol
\documentclass[11pt]{article}
\usepackage{cs70}

%%%%%%%%%%%%%%%%%%%% name/id
\rfoot{\small Zehao Huang | 3033857597 | zehao@berkeley.edu}


%%%%%%%%%%%%%%%%%%%% Course/HW info
\newcommand*{\instr}{Babak Ayazifar and Satish Rao}
\newcommand*{\term}{Spring 2019}
\newcommand*{\coursenum}{CS 70}
\newcommand*{\coursename}{Discrete Mathematics and Probability Theory}
\newcommand*{\hwnum}{07}
\newcommand{\Mod}[1]{\ (\mathrm{mod}\ #1)}
\renewcommand*{\bmod}{\mathbin{\%}}


%%%%%%%%%%%%%%%%%%%%%%%%%%%%%% Document Start %%%%%%%%%%%%%%%%%
\begin{document}
\section*{Sundry}

\begin{Answer}
    I typeset the homework using \LaTeX and discussed the problems with the following people.
    \begin{itemize}
        \item Yijia Chen -- yijia.chen@berkeley.edu
    \end{itemize}
\end{Answer}

\newpage
\Question{Rubik's Cube Scrambles}

\begin{Parts}
    \Part The $2\times2\times2$ Rubik's Cube is composed of 8 "corner pieces" arranged in a 
    $2\times2\times2$ cube. How many ways can we assign all the corner pieces a position?
    \begin{Answer}
        $8 \times 7 \times 6 \times \ldots \times 1 = 8!$
    \end{Answer}

    \Part Each corner piece has three distinct colors on it, and so can also be oriented three 
    different ways once it is assigned a position (see figure below). How many ways can we 
    \emph{assemble} (assign each piece a position and orientation) a $2\times2\times2$ Rubik's Cube?
    \begin{Answer}
        $(8 \times 3)(7 \times 3)(6 \times 3)\ldots(1 \times 3) = 8! \times 3^8$
    \end{Answer}

    \Part The previous part assumed we can take apart pieces and assemble them as we wish. 
    But certain configurations are unreachable if we restrict ourselves to just turning the 
    sides of the cube. What this means for us is that if the orientations of 7 out of 8 of the corner 
    pieces are determined, there is only 1 valid orientation for the eighth piece. Given this, how 
    many ways are there to \emph{scramble} (as opposed to \emph{assemble}) a $2\times2\times2$ Rubik's Cube?
    \begin{Answer}
        $(8 \times 3)(7 \times 3)(6 \times 3)\ldots 1 = 8! \times 3^7$
    \end{Answer}

    \Part We decide to treat scrambles that differ only in overall positioning (e.g. flipped 
    upside-down or rotated but otherwise unaltered) as the same scramble. Then we overcounted 
    in the previous part! How does this new condition change your answer to the previous part?
    \begin{Answer}
        $8! \times 3^7 / 12$
    \end{Answer}

    \Part Now consider the $3\times3\times3$ Rubik's Cube. In addition to 8 corner pieces, we 
    now have 12 "edge" pieces, each of which can take 2 orientations. How many ways can we 
    \emph{assemble} a $3\times3\times3$ Rubik's Cube?
    \begin{Answer}
        TODO: Finish this problem
    \end{Answer}

\end{Parts}

\newpage
\Question{Counting, Counting, and More Counting}

\begin{Parts}

    \Part How many ways are there to arrange $n$ 1s and $k$ 0s into a sequence?
    \begin{Answer}
        ${n + k \choose n}$
    \end{Answer}

    \Part A bridge hand is obtained by selecting 13 cards from a standard
    52-card deck. The order of the cards in a bridge hand is
    irrelevant. 
    \begin{enumerate}[(i)]
        \item How many different 13-card bridge hands are there? 
        \begin{Answer}
            ${52 \choose 13}$
        \end{Answer}

        \item How many different 13-card bridge hands are there that contain no aces? 
        \begin{Answer}
            ${48 \choose 13}$
        \end{Answer}

        \item How many different 13-card bridge hands are there that contain all four aces? 
        \begin{Answer}
            ${48 \choose 9}$
        \end{Answer}

        \item How many different 13-card bridge hands are there that contain exactly 6 spades?
        \begin{Answer}
            ${13 \choose 6} \times {39 \choose 7}$
        \end{Answer}
    \end{enumerate}

    \Part Two identical decks of 52 cards are mixed together, yielding a stack of 104 cards. 
    How many different ways are there to order this stack of 104 cards?
    \begin{Answer}
        ${\frac{104!}{2^{52}}}$
    \end{Answer}
    
    \Part How many 99-bit strings are there that contain more ones than zeros?
    \begin{Answer}
        $2^{98}$
    \end{Answer}
    
    \Part An anagram of FLORIDA is any re-ordering of the letters of FLORIDA, i.e., any string made up 
    of the letters F, L, O, R, I, D, and A, in any order. The anagram does not have to be an English word. 
    \begin{enumerate}[(i)]
        
        \item How many different anagrams of FLORIDA are there?
        \begin{Answer}
            $7!$
        \end{Answer}
        
        \item How many different anagrams of ALASKA are there?
        \begin{Answer}
            $\frac{6!}{3!}$
        \end{Answer}

        \item How many different anagrams of ALABAMA are there? 
        \begin{Answer}
            $\frac{7!}{4!}$
        \end{Answer}

        \item How many different anagrams of MONTANA are there?
        \begin{Answer}
            $\frac{7!}{2!2!}$
        \end{Answer}

    \end{enumerate}   
    
    \Part How many different anagrams of ABCDEF are there if: 
    \begin{enumerate}[(i)]
        
        \item C is the left neighbor of E; 
        \begin{Answer}
            $5!$
        \end{Answer}

        \item C is on the left of E (and not necessarily E's neighbor
        \begin{Answer}
            $\frac{6!}{2}$
        \end{Answer}

    \end{enumerate}

    \Part We have 9 balls, numbered 1 through 9, and 27 bins. How many different ways are there to distribute these 9 balls among
    the 27 bins? Assume the bins are distinguishable (e.g., numbered 1 through 27).
    \begin{Answer}
        $27^9$
    \end{Answer}
    
    \newpage
    \Part We throw 9 identical balls into 7 bins. How many different ways are there to distribute these 9 balls among
    the 7 bins such that no bin is empty? Assume the bins are distinguishable (e.g., numbered 1 through 7).
    \begin{Answer}
        ${8 \choose 2}$
    \end{Answer}
    
    \Part How many different ways are there to throw 9 identical balls into 27 bins? Assume the bins are distinguishable (e.g., numbered 1
    through 27).
    \begin{Answer}
        ${35 \choose 9}$
    \end{Answer}
    
    \Part There are exactly 20 students currently enrolled in a class. How many different ways are there to pair up the 20 students, so
    that each student is paired with one other student?
    \begin{Answer}
        ${\frac{19!}{9!2^9}}$
    \end{Answer}
    
    \Part How many solutions does $x_0 + x_1 + \cdots + x_k = n$ have, if each $x$ must be a non-negative integer?
    \begin{Answer}
        ${n+k \choose k}$
    \end{Answer}

    \Part How many solutions does $x_0 + x_1 = n$ have, if each $x$ must be a \emph{strictly positive} integer?
    \begin{Answer}
        $n-1$
    \end{Answer}
    
    \Part How many solutions does $x_0 + x_1 + \cdots + x_k = n$ have, if each $x$ must be a \emph{strictly positive} integer?
    \begin{Answer}
        $n - 1 \choose k$
    \end{Answer}

\end{Parts}

\newpage
\Question{Divisor Graph Colorings}

Define $G$ where we have $V=\{2, 3, 4, 5, 6, 7, 8, 9\}$, and we add an 
edge between vertex $i$ and vertex $j$ if $i$ divides $j$, or $j$ divides $i$.

\begin{Parts}
    
    \Part Explain why we cannot vertex-color G with only 2 colors.

    \begin{Answer}
        Consider vertices $2,4,8$. Since $2 \mid 4, 4 \mid 8, 2 \mid 8$, there 
        is a cycle among $2$, $4$, and $8$, which requires at least three colors. 
        Thus we need to vertex-color $G$ in at least three colors. 
    \end{Answer}

    \Part How many ways can we vertex-color G with 3 colors?

    \begin{Answer}
        We start coloring from vertex $2$. Assume there are colors $A,B,C$. 
        Suppose vertex $2$ is colored $A$, then there are two ways to color vertices
        $4,8$. For each coloring of the $2,4,8$ cycle, since none of them has any 
        connection to other nodes except $2$, node $6$ can choose $2$ colors, 
        node $3$ can choose two colors, and finally node $9$ can choose $2$ colors. 
        The total number of colorings for nodes $2,4,8,3,6,9$ is $6 \times 2^3 = 48$. 
        Since vertices $5,7$ are disconnected, they can be colored freely. Thus 
        the total number of coloring of the whole graph is $48 \times 3^2 = 432$. 
    \end{Answer}

\end{Parts}

\newpage
\Question{Vacation Time}
After a number of complaints, the Dunder Mifflin Paper Company has decided on the following rule for vacation leave for the next year (365 days): Every employee must take exactly one vacation leave of 4 consecutive days, one vacation leave of 5 consecutive days and one vacation leave of 6 consecutive days within the year, with the property that any two of the vacation leaves have a gap of at least 7 days between them. In how many ways can an employee arrange their vacation time? (The vacation policy resets every year, 
so there is no need to worry about leaving a gap between this year and next year's vacations).

\begin{Answer}
    The total number of vacation periods in the year is $3$. The total number of days an employee is not on 
    vacation is $365 - 4 - 5 - 6 = 350$. Besides every two vacation has to be apart from each other for 
    at least $7$ days. Thus for any ordering, we can imagine the first taking up $v_1 + 7$ days, the second 
    one taking up $v_2 + 7$ days, and the third one taking up $v_3$ days where $\{v_1, v_2, v_3\}$ is an 
    ordering of $\{4,5,6\}$. Thus the total number of days an employee not on vacation is $350 - 14 = 336$. 
    Imagine there are $339$ slots where three of them are vacations and $336$ of them are normal days. 
    The three vacations are distinguishable and the days are not. Thus we have the total number of ways of
    arranging the vacations as $339 \times 338 \times 337 = 38,614,134$. 
\end{Answer}

\newpage
\Question{Story Problems}

\newcommand{\sblank}{\vspace{1in}}
Prove the following identities by combinatorial argument:
\begin{Parts}

    \Part $\binom{2n}{2} = 2 \binom{n}{2} + n^2$
    
    \begin{Answer}
        \begin{proof}
            How many ways are there to choose $2$ elements from a size $2n$ set? \\
            Consider splitting the set with size $2n$ with two arbitrary sets with sizes $n$. 
            We denote the two sets as $A,B$. We have the below cases for the $2$ elements we choose. 
            \begin{itemize}
                \item They are both in $A$. There are thus ${n \choose 2}$ ways. 
                \item They are both in $B$. There are thus ${n \choose 2}$ ways. 
                \item One of them is in $A$ and the other is in $B$. There are $n \times n = n^2$ ways. 
            \end{itemize}
            Thus the total number of ways to choose $2$ items from a size $2n$ set is $2{n \choose 2} + n^2$. 
        \end{proof}
    \end{Answer}

  	\Part $n^2 = 2 \binom{n}{2} + n$

    \begin{Answer}
        \begin{proof}
            How many ways are there to choose 2 elements from a size $A$ set with replacement? \\
            Since there's replacement, both the first and the second has $n$ choices, so the answer
            is $n^2$. There are only two cases for the two elements. They are the same or they are different. 
            \begin{itemize}
                \item The two elements are the same. There are totally $n$ choices. 
                \item The two elements are different. It's the same as choosing $2$ without replacement except 
                that the order matters. So the total number of choices is $n(n - 1) = 2{n \choose 2}$. 
            \end{itemize}
            Thus the total number of choices for the two elements is $n + 2{n \choose 2} = n^2$. 
        \end{proof}
    \end{Answer}

    \Part $\sum_{k=0}^n k {n \choose k} = n2^{n-1}$

    \begin{Answer}
        \begin{proof}
            How many ways are there to pick a subset of people from $n$ people and then choose a leader? 
            The subset of people can have sizes $0$ to $n$. For each possible size $k$ we have ${n \choose k}$ 
            ways to form the subset. To choose a leader from each subset, we can have $k$ ways. Thus the 
            total number of possibilities for this problem is $\sum_{k=0}^n k{n \choose k}$. \\
            Consider the problem from another perspective. Each person can be a leader, so there are $n$ 
            possible leaders. For each leader determined, there are $n-1$ people left to choose. Any person 
            can be chosen in the subset or not. So the total number of possible subsets for each leader 
            is $2^{n-1}$. With $n$ possible leaders, there are $n2^{n-1}$ possibilities to this problem. 
        \end{proof}
    \end{Answer}

    \Part $\sum_{k=j}^n {n \choose k} {k \choose j} = 2^{n-j} {n \choose j}$ 
    
    \begin{Answer}
        How many ways are there to choose $k$ people from $n$ people and then choose $j$ people from the 
        $k$ people? \\
        The first way to consider this problem is to consider the group of $k$ people first. In order to 
        choose $j$ people from it, $n \geq k \geq j$. For each possible size $k$, we have ${n \choose k}$
        ways to form the group. Then for each group there are ${k \choose j}$ ways to finalize the $j$ 
        people. Thus the answer to the problem is $\sum_{k=j}^n{n \choose k}{k \choose j}$. \\
        Another way to consider this problem is to consider the $j$ people chosen. There are in total 
        ${n \choose j}$ ways to choose the $j$ people directly. For each possible $j$ people group, there 
        are $n - j$ people left. Each person among these $n - j$ people can either belong to the 
        $k$-people group or not. Thus there are $2^{n-j}$ ways to form the $k$-people group the 
        that the $j$-people group belongs to for each $j$-people group. Thus the answer to this 
        problem can also be written as $2^{n-j}{n \choose j}$. \\
        Thus we conclude that $\sum_{k=j}^n {n \choose k} {k \choose j} = 2^{n-j} {n \choose j}$. 
    \end{Answer}

\end{Parts}

\newpage
\Question{Fermat's Wristband}

Let $p$ be a prime number and let $k$ be a positive integer.
We have beads of
$k$ different colors, where any two beads of the same color are indistinguishable.

\begin{Parts}

    \Part
    We place $p$ beads onto a string.
    How many different ways are there construct such a sequence of $p$ beads with up to $k$ different colors?
    \begin{Answer}
        Each bead among the $p$ beads have $k$ choices of colors. Thus the total number of ways 
        to construct a sequence of $p$ beads is $k^p$.
    \end{Answer}

    \Part 
    How many sequences of $p$ beads on the string are there that use at least two colors?
    \begin{Answer}
        The total number to use one color to construct the sequence is $k$. So the answer is $k^p - k$. 
    \end{Answer}

    \Part
    Now we tie the two ends of the string together, forming a wristband. Two wristbands are 
    equivalent if the sequence of colors on one can be obtained by rotating the beads on the other.
    (For instance, if we have $k=3$ colors, red (R), green (G), and blue (B), then the length $p = 5$
    necklaces RGGBG, GGBGR, GBGRG, BGRGG, and GRGGB are all equivalent, because these are all rotated 
    versions of each other.) How many non-equivalent wristbands are there now? Again, the $p$
    beads must not all have the same color.
    \begin{Answer}
        Under the new condition, consider rotating one necklaces $p - 1$ times. Combined with the 
        original neckless, the $p$ neckless are considered different in our original calculation 
        but are considered as $1$ unique neckless in this problem. So the answer is $\frac{k^p - k}{p}$. 
    \end{Answer}

    \Part Use your answer to part (c) to prove Fermat's little theorem.

    \begin{Answer}
        \begin{proof}
            The answer to the last problem must be an integer, so $p \mid k^p - k$. We also have 
            $k^p - k = k(k^{p - 1} - 1)$. 
            \begin{align*}
                p \mid k^p - k \implies k^p \equiv k \Mod{p}
            \end{align*}
            Thereby Fermat's Little Theorem is proved. 
        \end{proof}
    \end{Answer}

\end{Parts}

\end{document}