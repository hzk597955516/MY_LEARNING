%Template compiled by Alice lol
\documentclass[11pt]{article}
\usepackage{cs70}

%%%%%%%%%%%%%%%%%%%% name/id
\rfoot{\small Zehao Huang | 3033857597 | zehao@berkeley.edu}


%%%%%%%%%%%%%%%%%%%% Course/HW info
\newcommand*{\instr}{Babak Ayazifar and Satish Rao}
\newcommand*{\term}{Spring 2019}
\newcommand*{\coursenum}{CS 70}
\newcommand*{\coursename}{Discrete Mathematics and Probability Theory}
\newcommand*{\hwnum}{03}
\newcommand{\Mod}[1]{\ (\mathrm{mod}\ #1)}
\renewcommand*{\bmod}{\mathbin{\%}}


%%%%%%%%%%%%%%%%%%%%%%%%%%%%%% Document Start %%%%%%%%%%%%%%%%%
\begin{document}
\section*{Sundry}

\begin{Answer}
    I typeset the homework using \LaTeX and discussed the problems with the following people. 
    \begin{itemize}
        \item Yijia Chen -- yijia.chen@berkeley.edu
    \end{itemize}
\end{Answer}

\newpage
\question{Leaves in a Tree}

\begin{Parts}
    \Part What is the largest possible number of leaves the tree could have?
    \begin{Answer}
        The largest possible number of leaves a tree could have is $n-1$. \\
        \textbf{Claim:} $m=n-1$ is achievable, and there cannot exist a tree with more than $m$ leaves.
        \begin{proof}
            The proof will be by the induction on $n$. \\
            \emph{Base Case:} $n=3$. $V=\{v_1,v_2,v_3\}$, $E=\{(v_1,v_2),(v_1,v_3)\}$. $G=(V,E)$ has 2 leaves. \\
            \emph{Inductive Hypothesis:} $n=k$, $m=k-1$ is achievable and no tree has more than $m$ leaves. \\
            \emph{Inductive Step:} $n=k+1$, we consider the original graph $G_k$ and insert the $k+1^{th}$ vertex. 
                                   For any graph, the total degree of vertices is $2|E|$. The $k-1$ leaves in $G_k$
                                   take up $k-1$ degrees, leaving the remaining vertex $v$ a degree of $k-1$. For the
                                   $k+1^{th}$ vertex, we use $1$ edge to connect it to $v$ so that it becomes another 
                                   leave. Now $G_{k+1}$ has $k$ leaves because it has one more leaf that $G_k$. If there
                                   are more than $k$ leaves, there can only be $k+1$ leaves because the total number of 
                                   vertices is $k+1$. However, all nodes are leaves implies that the total degree of 
                                   the graph is $k+1$. Obviously $k+1 \neq 2k$ when $k \geq 3$. Thus we have contradiction.\\
            We conclude that $m=n-1$ is achievable, and no tree with $n$ vertices has more than $m$ leaves. 
        \end{proof}
    \end{Answer}

    \Part Prove that every tree on $n\geq 2$ vertices must have at least 2 leaves. 
    \begin{Answer}
        \textbf{Claim:} Every tree on $n\geq 2$ vertices must have at least 2 leaves. 
        \begin{proof}
            We will prove the claim by contradiction. \\ 
            Assume that some tree has less than $2$ leaves. According to the definition of trees, $|E|=|V|-1$ and for any
            graph, the sum of degrees of nodes is $2|E|=2|V|-2=2n-2$. The tree can have either $1$ or $0$ leaves.
            \begin{itemize}
                \item The tree has $1$ leaf. The sum of degrees of the tree is at least $1+2(n-1)=2n-1$. 
                \item The tree has $0$ leaf. The sum of degrees of the tree is at least $2n$. 
            \end{itemize}
            Both circumstances contradicts the fact that the total degrees of vertices is $2n-2$. Thus by contradiction
            we conclude that every tree on $n \geq 2$ vertices must have at least 2 leaves.  
        \end{proof}
    \end{Answer}
\end{Parts}

\newpage
\question{Coloring Trees}

\begin{Parts}
    \Part What is the minimum number of colors needed to color a tree? Prove it. 
    \begin{Answer}
        \textbf{Claim:} The minimum number of colors needed to color a tree is $2$. 
        \begin{proof}
            The proof will be by the induction on $n$. \\
            \emph{Base Case:} $n \leq 2$, the tree can be trivially colored by $2$ colors. \\
            \emph{Inductive Hypothesis:} Any tree with $k$ vertices can be two-colored. \\
            \emph{Inductive Step:} We must show that any tree $G$ with $k+1$ vertices can be two-colored. \\
            We have proved in question $1$ that every tree has at least two leaves. For any tree on $k+1$
            vertices, we choose a leaf $v$ and its adjacent vertex $v^*$. We color $v$ blue and $v^*$ red.
            Then we remove $v$ from $G$. It can be seen that $G-v$ is still a tree because we have removed 
            exactly one vertex and one edge and $|V|=|E|+1$ still holds for $G-v$. Thus according to the 
            inductive hypothesis we can color $G-v$ using two colors and we just color $v$ and $v^*$ differently. \\
            Thus we conclude that the minimum number of colors needed to color a tree is $2$. 
        \end{proof}
    \end{Answer}

    \Part Prove that all trees are bipartite. 
    \begin{Answer}
        \textbf{Claim:} All trees are bipartite. 
        \begin{proof}
            We will prove the claim directly. \\
            Since we have proved in part (a) that all trees can be two-colored, we assume that we have colored
            a tree on $n$ vertices using blue and red. We group all blue vertices into one group and all red
            vertices into another group. For the coloring to be valid, there does not exists an edge $e(u,v)$
            such that both $u,v$ are colored blue or both $u,v$ are colored red. That is to say, no edge connects
            any two node in the same group. All edges in the tree goes from the blue group to the red group. \\
            We conclude that all trees are bipartite. 
        \end{proof}
    \end{Answer}
\end{Parts}

\newpage
\question{Edge-Disjoint Paths in a Hypercube}

\begin{Answer}
    \textbf{Claim:} Between any two distinct vertices in the $n$-dimensional hypercube, there are at least $n$ 
    edge-disjoint paths from $x$ to $y$.
    \begin{proof}
        The proof will be by the induction on $n$. \\
        \emph{Base Case:} $n=2$, $G$ is a square. We can go left and right from any vertex to reach the target. \\
        \emph{Inductive Hypothesis:} Assume there are at least $k$ edge-disjoint paths from $x$ to $y$ ($x \neq y$). \\
        \emph{Inductive Step:} We must prove that there are at least $k+1$ edge-disjoint paths from $x$ to $y$ ($x \neq y$) 
        in a $k+1$-dimensional hypercube. According to the recursive definition of hypercubes, one of the below two cases must hold. 
        \begin{itemize}
            \item $x,y$ belongs to the same $k$-dimensional hypercube. \\
                  According to the inductive hypothesis, there are at least $k$ edge-disjoint paths from $x$ to $y$ 
                  within the same hypercube they belong to. Since the $k+1$ dimensional hypercube is defined by 
                  building an edge between each pair of corresponding vertices in two sub-hypercubes, we can define 
                  the corresponding vertices of $x,y$ in the other hypercube as $x^*,y^*$, and it is trivial that 
                  there is at least one path from $x^*$ to $y^*$. Thus, one additional path from $x$ to $y$ is formed
                  by going from $x$ to $x^*$ and pick a path from $x^*$ to $y^*$ and then go from $y^*$ to $y$. This
                  path does not pass any edge on the hypercube that $x,y$ belong to; as a result, it is a valid 
                  edge-disjoint path compared to the original $k$ edge-disjoint paths within the $k$-dimensional hypercube. 
                  We have by far found at least $k+1$ edge-disjoint paths from $x$ to $y$. 
            \item $x,y$ belongs to different $k$-dimensional hypercubes. \\
                  Assume that $x$ belongs to sub-hypercube $H_1$ and $y$ belongs to $H_2$. Let $x$'s corresponding vertex
                  on $H_2$ be $x^*$ and $y$'s corresponding vertex on $H_1$ be $y^*$. 
                  \begin{itemize}
                      \item $x^*=y$. $x$ has $k$ neighbors in $H_1$. $k$ edge-disjoint paths can be found by first going 
                            to those neighbors, then go to their corresponding vertices in $H_2$, then converge back to $y$. The
                            $k+1$th path is the edge connecting $x$ and $x^*$, which is $y$.
                      \item $x^*\neq y$. We first map out the $k$ paths from $x$ to $y^*$ within $H_1$. Our final goal is to arrive 
                            at $y$. We first find the $k+1^{th}$ path by sending $x$ to $x^*$ and build a path from $x^*$ to $y$. 
                            In the rest $k$ paths, there must be one that can be mirrored to that from $x^*$ to $y$. For that path 
                            we finish the mirror path in $H_1$ from $x$ to $y^*$ and then send $y^*$ to $y$. For the rest $k-1$ paths,
                            we send $x$'s neighbors to their corresponding nodes and finish the mirror path on $H_2$. 
                  \end{itemize}
        \end{itemize}
        We can now conclude that between any two distinct vertices in the $n$-dimensional hypercube, there are at least $n$ 
        edge-disjoint paths from $x$ to $y$.
    \end{proof}
\end{Answer}

\newpage
\question{Triangulated Planar Graph}

\begin{Parts}
    \Part What is the sum of the charges on all the vertices? 
    \begin{Answer}
        12. According to the definition of the graph, each face $f$ is surrounded by exactly $3$ edges. Since each edge touches 
        on two faces exactly, we have $e=3f/2$. Since this is a planar graph, we also have $e+2=f+v$. Thus $f=2v-4$, $e=3v-6$. 
        The sum of degrees of all vertices is $2e$. Thus the total sum of charges is $6v-2e=6v-6v+12=12$.
    \end{Answer}

    \Part What is the charge of a degree $5$ vertex and of a degree $6$ vertex. 
    \begin{Answer}
        The charge of a degree $5$ vertex is $1$. \\
        The charge of a degree $6$ vertex is $0$. 
    \end{Answer}
    
    \Part After discharging all degree $5$ vertices, there's a degree $5$ vertex with positive remaining charge. 
    \begin{Answer}
        \begin{proof}
            Assume a degree $5$ vertex with a positive remaining charge is $u$. The assumption above implies that
            there exists a neighbor $v$ such that charge$(v) \geq 0$, i.e. degree$(v)$ $\leq 6$.
            \begin{itemize}
                \item degree$(v)=6$, a degree 5 and a degree 6 vertices are adjacent. 
                \item degree$(v)=5$, two degree 5 vertices are adjacent. 
                \item degree$(v)<5$, there exists a vertex of degree $1,2,3,$ or $4$. 
            \end{itemize}
            We conclude the claim in the prompt.
        \end{proof}
    \end{Answer}

    \Part No degree $5$ vertex with positive remaining charge. Any $v$ with positive charge? Possible degrees?
    \begin{Answer}
        We enumerate the below possible cases. 
        \begin{itemize}
            \item It's simply possible for degree$(v)<5$.
            \item It's impossible for degree$(v)=6$ because $v$ is not affected during discharging and charge$(v)$ remains to be $0$. 
            \item It's also impossible for degree$(v)>7$.  At least $1-5\cdot$charge$(v)$ degree $5$ neighbors are 
                  required for $v$ to receive enough charge. $1-5\cdot$charge$(v)=5\cdot$degree$(v)-29>$ degree$(v)$ when $v>7$. 
                  There will never be enough neighbors for $v$ to gain enough charge.
        \end{itemize}
        Thus we conclude that degree$(v)$ is $1,2,3,4$ or $7$. 
    \end{Answer}

    \newpage
    \Part Suppose there exists a degree 7 vertex with positive charge after discharging the degree 5 vertices. How many neighbors of degree 5 might it have?
    \begin{Answer}
        It has 6 or 7 neighbors of degree 5 vertices. 
    \end{Answer}

    \Part Are two of these degree 5 vertices adjacent?
    \begin{Answer}
        For the graph to be triangulated, there must be a cycle that connects all neighbors of $v$. That means
        there exists and edge $e(x,y)$ such that degree$(x)=$ degree$(y)=5$. When discharging, $x$ and $y$ 
        give out at most $4$ fifths of their charges because they don't give out their charges to each other. 
        This contradicts that no degree 5 vertex has a positive remaining charge. 
    \end{Answer}

    \Part Finish the proof from the facts you obtained from the previous parts.
    \begin{Answer}
        \begin{proof}
            We will prove the claim by cases.\\
            After discharging all degree 5 vertices, one of the below two cases must hold. 
            \begin{itemize}
                \item There's a degree $5$ vertex with positive remaining charge. \\
                      It has been shown in part (c) that the claim holds. 
                \item There's no degree $5$ vertex with positive remaining charge. \\
                      There must exist some vertex $v$ whose charge is positive because the total sum of charges 
                      of all vertices remains to be 12 after discharging. From part (d) we know that the possible 
                      degrees of $v$ are $1,2,3,4,7$. However, from parts (e) and (f) we learn that the degree of $v$
                      cannot be 7 in order for the graph to be triangulated. We now know that there exists a $v$ 
                      whose degree is $1,2,3,4$. 
            \end{itemize}
            The claim holds after combining the two cases. Thus we conclude that every triangulated planar graph 
            contains either (1) a vertex of degree 1, 2, 3, 4, (2) two degree 5 vertices which are adjacent, or 
            (3) a degree 5 and a degree 6 vertices which are adjacent.
        \end{proof}
    \end{Answer}
\end{Parts}

\newpage
\question{Euclid's Algorithm}

\begin{Parts}
    \Part $\gcd(527,323)$.
    \begin{Answer}
        \begin{align*}
            \gcd(527,323) &= \gcd(323,527\bmod 323) = \gcd(323,205) \\
                          &= \gcd(205,323\bmod 205) = \gcd(205,118) \\
                          &= \gcd(118,205\bmod 118) = \gcd(118,87) \\
                          &= \gcd(87,118\bmod 87) = \gcd(87,31) \\
                          &= \gcd(31,87\bmod 31) = \gcd(31,25) \\
                          &= \gcd(25,31\bmod 25) = \gcd(25,6) \\
                          &= \gcd(6,25\bmod 6) = \gcd(6,1) \\
                          &= 1
        \end{align*}
    \end{Answer}

    \Part Find multiplicative inverse of $5\Mod{27}$. 
    \begin{Answer}
        This is the same as solving $27x+5y=1$ using extended Euclid's Algorithm. 
        \begin{align*}
            \gcd(27,5) &= \gcd(5,2) = \gcd(2,1) = \gcd(1,0) \\
            \gcd(1,0)  &= (1, 1, 0) \\
            \gcd(2,1)  &= (1, 0, 1) \\
            \gcd(5,2)  &= (1, 1, -2) \\
            \gcd(27,5) &= (1, -2, 11)
        \end{align*}
        $x=-2,y=11$. We now know that the multiplicative inverse of $5\Mod{27}$ is $11$. 
    \end{Answer}

    \Part Find $x \pmod{27}$ if $5x + 26 \equiv 3 \pmod{27}$. 
    \begin{Answer}
        \begin{align*}
            5x + 26 &\equiv 3 \Mod{27} \\
            5x - 1  &\equiv 3 \Mod{27} \\
            5x      &\equiv 4 \Mod{27} \\
            x       &\equiv 4\times 11 \Mod{27} \\
            x       &\equiv 17 \Mod{27}
        \end{align*}
    \end{Answer}

    \Part If $a$ has no multiplicative inverse mod $c$, then $ax \equiv b \pmod{c}$ has no solution.
    \begin{Answer}
        The claim is false. We disprove by giving counterexamples to the claim. \\
        Suppose $a=3,b=6,c=9$. $a$ doesn't have a multiplicative inverse $\Mod{c}$ since $\gcd(a,c)=3$. 
        However, $x=2$ is a solution to the equation $ax\equiv b \Mod{c}$. 
    \end{Answer}
\end{Parts}

\newpage
\question{Fibonacci GCD}

\begin{Answer}
    \textbf{Claim:} $F_0=0$, $F_1=1$.$F_n=F_{n-1}+F_{n-2}$. $\forall n\geq 0$, $\gcd(F_n,F_{n-1})=1$.
    \begin{proof}
        The proof will be by the induction on $n$. \\
        \emph{Base Case:} $n=1$, $\gcd(F_0,F_1)=\gcd(0,1)=1$. \\
        \emph{Inductive Hypothesis:} $n\geq 2$, assume $\gcd(F_{n-1},F_{n-2})=1$ is true. \\
        \emph{Inductive Step:} We must show that $\gcd(F_n,F_{n-1})=1$ is true. \\ 
        $F_{n-2}<F_{n-1} \implies F_n=F_{n-2}+F_{n-1}<2F_{n-1}$. \\
        $2F_{n-1}>F_n>F_{n-1} \implies \floor{F_n/F_{n-1}}=1 \implies F_n\bmod F_{n-1}=F_n-F_{n-1}$. \\
        According to Euclid's Algorithm: 
        \begin{align*}
            \gcd(F_n,F_{n-1}) &= \gcd(F_{n-1},F_n\bmod F_{n-1}) \\
                              &= \gcd(F_{n-1},F_n-F_{n-1}) \\
                              &= \gcd(F_{n-1},F_{n-2}) \\
                              &= 1
        \end{align*}
        We conclude that in a Fibonacci sequence, for all $n \geq 0$, $\gcd(F_n, F_{n-1})=1$.
    \end{proof}
\end{Answer}
\end{document}