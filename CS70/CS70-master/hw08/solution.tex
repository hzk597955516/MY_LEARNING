%Template compiled by Alice lol
\documentclass[11pt]{article}
\usepackage{cs70}

%%%%%%%%%%%%%%%%%%%% name/id
\rfoot{\small Zehao Huang | 3033857597 | zehao@berkeley.edu}


%%%%%%%%%%%%%%%%%%%% Course/HW info
\newcommand*{\instr}{Babak Ayazifar and Satish Rao}
\newcommand*{\term}{Spring 2019}
\newcommand*{\coursenum}{CS 70}
\newcommand*{\coursename}{Discrete Mathematics and Probability Theory}
\newcommand*{\hwnum}{07}
\newcommand{\Mod}[1]{\ (\mathrm{mod}\ #1)}
\renewcommand*{\bmod}{\mathbin{\%}}


%%%%%%%%%%%%%%%%%%%%%%%%%%%%%% Document Start %%%%%%%%%%%%%%%%%
\begin{document}
\section*{Sundry}

\begin{Answer}
    I typeset the homework using \LaTeX and discussed the problems with the following people.
    \begin{itemize}
        \item Yijia Chen -- yijia.chen@berkeley.edu
    \end{itemize}
\end{Answer}

\newpage
\Question{Five Coins}

\begin{Parts}
    
    \Part For the first three parts, order matters in the outcome. How many different outcomes are possible?
    \begin{Answer}
        $2^5 = 32$ different outcomes. 
    \end{Answer}

    \Part How many different outcomes are possible with exactly 3 heads?
    \begin{Answer}
        ${5 \choose 3} = 10$ different outcomes. 
    \end{Answer}

    \Part How many different outcomes are possible with 3 or more heads? Justify with an argument.
    \begin{Answer}
        ${32 / 2 = 16}$ different outcomes. If we want $3$ or more heads, we can have $3$, $4$, or $5$ heads. The 
        complement to what we want in the sample space are those combinations with $3$ or more tails. Since the 
        coin is unbiased, these two situations should be symmetric and thus have the same numbers of outcomes. 
    \end{Answer}

    \Part What is the probability of getting the outcome TTHHH? What is the probability of getting the outcome THHHH?
    \begin{Answer}
        The probability for TTHHHH is $\frac{1}{32}$. The probability for THHHH is $\frac{1}{32}$.
    \end{Answer}

    \Part What’s the probability of getting at least one heads?
    \begin{Answer}
        The probabilities for getting at least one heads is $1 - \frac{1}{32} = \frac{31}{32}$. 
    \end{Answer}

    \Part What’s the probability of getting 3 or more heads?
    \begin{Answer}
        According to the symmetry mentioned in $c$, the probability is $\frac{1}{2}$. 
    \end{Answer}

    \newpage
    \Part For the next three parts, assume that the coin is biased with probability of heads being $\frac{2}{3}$. 
          What is the probability of getting the outcome TTHHH? What is the probability of getting the outcome THHHH?
    \begin{Answer}
        The probability for TTHHH is $(\frac{1}{3})^2 \times (\frac{2}{3})^3 = \frac{8}{243}$. \\
        The probability for THHHH is $\frac{1}{3} \times (\frac{2}{3})^4 = \frac{16}{243}$.  
    \end{Answer}
    
    \Part What’s the probability of getting at least one heads?
    \begin{Answer}
        The probability of getting at least one heads is $1 - (\frac{1}{3})^5 = \frac{242}{243}$. 
    \end{Answer}

    \Part What’s the probability of getting 3 or more heads?
    \begin{Answer}
        The probability of getting $3$ heads is ${5 \choose 3}\frac{8}{243} = \frac{80}{243}$  \\
        The probability of getting $4$ heads is ${5 \choose 4}\frac{16}{243} = \frac{80}{243}$ \\
        The probability of getting $5$ heads is ${5 \choose 5}\frac{32}{243} = \frac{32}{243}$ \\
        Thus the probability of getting 3 or more heads is $\frac{192}{243} = \frac{64}{81}$. 
    \end{Answer}

\end{Parts}

\newpage
\Question{Weathermen}

\begin{Parts}
    
    \Part If Tom says that it is going to snow, what is the probability it will actually snow?
    \begin{Answer}
        We model the problem as follows. Let $A$ be the event that New York snows. Let $B$ be the event that Tom 
        says it will snow. According to the predicate, we have the following data. 
        \begin{align*}
            \mathbb{P}(A)                  &= 0.1 \\
            \mathbb{P}(B \mid A)           &= 0.7 \\
            \mathbb{P}(B \mid \overline A) &= 0.05
        \end{align*}
        Thus, $\mathbb{P}(A \mid B) = \frac{\mathbb{P}(A)\mathbb{P}(B \mid A)}{\mathbb{P}(B)} = \frac{0.07}{0.115} \approx 0.61$. 
    \end{Answer}

    \Part What is Tom’s overall accuracy?
    \begin{Answer}
        $\mathbb{P}(A)\mathbb{P}(B \mid A) + \mathbb{P}(\overline A)\mathbb{P}(\overline B \mid \overline A) = 0.925$. 
    \end{Answer}

    \Part Tom’s friend Jerry is a weatherman in Alaska. Jerry claims that she is a better weatherman than Tom even 
          though her overall accuracy is lower. After looking at their records, you determine that Jerry is indeed 
          better than Tom at predicting snow on snowy days and sun on sunny day. Give an instance of the situation 
          described above.
    \begin{Answer}
        Say Jerry correctly predicts the snow $80\%$ of the time and no snow $100\%$ of the time. If Alaska snows
        $90\%$ of the time, we have the overall accuracy of Jerry as $0.8 \times 0.9 + 1 \times 0.1 = 0.82 < 0.925$. 
        In this case Jerry is a better weatherman even though her overall accuracy is lower. 
    \end{Answer}

\end{Parts}

\newpage
\Question{Faulty Lightbulbs}

\begin{Parts}
    
    \Part Suppose a box is given to you at random and you randomly select a lightbulb from the box. If that lightbulb 
          is defective, what is the probability you chose Box 1?
    \begin{Answer}
        $\frac{0.5 \times 0.1}{0.5 \times 0.1 + 0.5 \times 0.05} \approx 0.67$. 
    \end{Answer}

    \Part Suppose now that a box is given to you at random and you randomly select two light- bulbs from the box. If 
    both lightbulbs are defective, what is the probability that you chose from Box 1?
    \begin{Answer}
        If we are given box $1$, the probability that both lightbulbs are defective is $\frac{100 \times 99}{1000 \times 999}$.
        If we are given box $2$, it's $\frac{100 \times 99}{2000 \times 1999}$. Thus given that both lightbulbs are defective, 
        the probability that we chose from box $1$ is $\frac{2000 \times 1999}{1000 \times 999 + 2000 \times 1999} \approx 0.80$. 
    \end{Answer}

\end{Parts}

\newpage
\Question{Solve the Rainbow}

\begin{Parts}
    
    \Part If you take a Skittle from the bag, what is the probability that it is green?
    \begin{Answer}
        If all green Skittles are eaten, the probability to take a green Skittle is $0$. \\
        If $10$ green Skittles are eaten, the probability to take a green Skittle is $\frac{1}{9}$. \\
        If $5$ green Skittles are eaten, the probability to take a green Skittle is $\frac{3}{19}$. \\
        Combining the result, the probability that it's green is $\frac{1}{4}(\frac{1}{9} + \frac{3}{19}) = \frac{23}{342}$. 
    \end{Answer}

    \Part If you take two Skittles from the bag, what is the probability that at least one is green?
    \begin{Answer}
        If all green Skittles are eaten, the probability is $0$. \\
        If $10$ green Skittles are eaten, the probability is $1 - \frac{{80 \choose 2}}{{90 \choose 2}} $. \\
        If $5$ green Skittles are eaten, the probability is $1 - \frac{{80 \choose 2}}{{95 \choose 2}}$. \\
        The result is $\frac{1}{4}(\frac{169}{801} + \frac{261}{893})$. 
    \end{Answer}

    \Part If you take three Skittles from the bag, what is the probability that they are all green?
    \begin{Answer}
        If all green Skittles are eaten, the probability is $0$. \\
        If $10$ green Skittles are eaten, the probability is $\frac{10 \times 9 \times 8}{90 \times 89 \times 88}$. \\
        If $5$ green Skittles are eaten, the probability is $\frac{5 \times 4 \times 3}{95 \times 94 \times 93}$. \\
        The result is $\frac{1}{4}(\frac{720}{704880} + \frac{60}{830490})$. 
    \end{Answer}

    \Part If all three Skittles you took from the bag are green, what are the probabilities that your roommate had all 
          of the green ones, half of the green ones, or only 5 green ones?
    \begin{Answer}
        Probability that all green Skittles are eaten is $0$. 
        Probability that $10$ green Skittles are eaten is $\frac{\frac{720}{704880}}{\frac{720}{704880} + \frac{60}{830490}}$.
        Probability that $5$ green Skittles are eaten is $\frac{\frac{60}{830490}}{\frac{720}{704880} + \frac{60}{830490}}$.
    \end{Answer}

    \Part If you take three Skittles from the bag, what is the probability that they are all the same color?
    \begin{Answer}
        The probability that they are all green is $\frac{1}{4}(\frac{720}{704880} + \frac{60}{830490})$. 
        The probability that they are all some other color is $4(\frac{1}{2}\frac{20 \times 19 \times 18}{80 \times 79 \times 78}$
        $+ \frac{1}{4}\frac{20 \times 19 \times 18}{90 \times 89 \times 88} + \frac{1}{4}\frac{20 \times 19 \times 18}{95 \times 94 \times 93})$.
        The sum is the total probability. 
    \end{Answer}

\end{Parts}

\newpage
\Question{Playing Strategically}

\begin{Parts}
    
    \Part Compute the probability of the event $E_1$ that Bob wins in a duel against Eve alone, assuming he shoots first.
    \begin{Answer}
        Let $x$ be the probability Bob wins in a duel against Eve alone. The probability that Bob directly hits Eve is 
        $\frac{1}{3}$. The probability that Bob misses Eve but still wins is $\frac{2}{3}\frac{1}{3}x$. For each arbitrary 
        round these are the only two possible cases for Bob to win, thus we have $\frac{1}{3} + \frac{2}{9}x = x$. Solving 
        the equation we have $x = \frac{3}{7}$. 
    \end{Answer}

    \Part Compute the probability of the event $E_2$ that Bob wins in a duel against Eve alone, assuming he shoots second.
    \begin{Answer}
        Let $x$ be the probability Bob wins in a duel against Eve alone assuming he shoots second. If Bob hits Eve in his 
        first shot, it means that Eve has to miss her first shot; thus the probability is $\frac{1}{3}\frac{1}{3}=\frac{1}{9}$.
        If Bob misses Eve, the probability that he wins is $\frac{1}{3}\frac{2}{3}x$. For each arbitrary round these are 
        the only two possible cases for Bob to win if he shoots second, thus we have $\frac{2}{9}x + \frac{1}{9} = x$. 
        Solving the equation we have $x = \frac{1}{7}$. 
    \end{Answer}

    \Part Compute the probability of the same events for a duel of Bob against Carol.
    \begin{Answer}
        Let $x$ be the probability Bob wins in a duel against Carol assuming he shoots first. The probability that Bob 
        hits Carol in his first shot is $\frac{1}{3}$. If he misses he will die for sure because Carol never misses. 
        Thus we have $x = \frac{1}{3}$ if Bob shoots first. \\
        Let $x$ be the probability Bob wins in a duel against Carol assuming he shoots second. The probability that Bob 
        hits Carol in his second shot is $0$ because Carol will never miss her first shot. Thus $x = 0$ if Bob shoots 
        second in a duel with Carol. 
    \end{Answer}

    \newpage
    \Part Assuming that both Eve and Carol play rationally, conclude that Bob’s best course of action is to shoot into 
          the air (i.e., intentionally miss)!
    \begin{Answer}
        Since Bob shoots first, we have the below three cases for Bob. 
        \begin{itemize}
            \item Bob hits Carol. Then Bob goes into a duel with Eve, shooting second, winning rate $\frac{1}{7}$. 
            \item Bob hits Eve. Then Bob goes into a duel with Carol, shooting second, winning rate $0$. 
            \item Bob misses. First we consider Carol. Since she never misses, her optimal strategy is to eliminate 
                  someone who posts more danger to her. In this case it's Eve, who shoots more precisely than Bob. 
                  Since Eve knows that if she doesn't shoot Carol, she will definitely be killed by Carol in the next 
                  round, in her round she will choose to shoot Carol. If she hits Carol, it's Bob's duel with Eve with 
                  Bob shooting first, winning rate $\frac{3}{7}$. If she misses Carol, Eve kills Carol and Bob goes into 
                  a duel with Carol with Bob shooting first, winning rate $\frac{1}{3}$. Thus if Bob misses his winning 
                  rate is $\frac{3}{7}\frac{2}{3} + \frac{1}{3}\frac{1}{3} = \frac{25}{63} > \frac{1}{7} > 0$. 
        \end{itemize}
        Thus we conclude that Bob's optimal strategy is to intentionally miss his first shoot.
    \end{Answer}

\end{Parts}

\newpage
\Question{Minesweeper}

\begin{Parts}
    
    \Part What is the probability that the square reveals a mine? a blank space? the number $k$? 
    \begin{Answer}
        The probability that the square reveals a mine is $\frac{5}{32}$. \\
        The probability that the square reveals a blank space is $\frac{{55 \choose 10}}{{64 \choose 10}}$. \\
        The probability that the square reveals $k$ is $\frac{{8 \choose k}{55 \choose 10-k}}{{64 \choose 10}}$.
    \end{Answer}

    \Part The first square you picked revealed the number k. For your next move, you want to minimize the probability of 
          picking a mine. Should you pick a square adjacent to your first pick, or a different square?
    \begin{Answer}
        The Probability that a square not adjacent to the first pick is a mine is $\frac{10 - k}{55}$. The probability that 
        a square adjacent to the first pick is a mine is $\frac{k}{8}$. 
        \begin{align*}
            \frac{10-k}{55} - \frac{k}{8} = \frac{80-63k}{440} = 0 \implies k = \frac{80}{63} 
        \end{align*}
        Thus if $k \leq 1$, choose an adjacent square. If $k > 1$, choose a non-adjacent square. 
    \end{Answer}

    \Part Your first move resulted in the number 1. You pick the square to the right for your next move. What is the 
          probability that this square reveals the number 4?
    \begin{Answer}
        I don't know. 
    \end{Answer}
\end{Parts}

\end{document}