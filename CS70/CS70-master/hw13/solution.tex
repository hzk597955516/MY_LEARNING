%Template compiled by Alice lol
\documentclass[11pt]{article}
\usepackage{cs70}

%%%%%%%%%%%%%%%%%%%% name/id
\rfoot{\small Zehao Huang | 3033857597 | zehao@berkeley.edu}


%%%%%%%%%%%%%%%%%%%% Course/HW info
\newcommand*{\instr}{Babak Ayazifar and Satish Rao}
\newcommand*{\term}{Spring 2019}
\newcommand*{\coursenum}{CS 70}
\newcommand*{\coursename}{Discrete Mathematics and Probability Theory}
\newcommand*{\hwnum}{13}
\newcommand{\Mod}[1]{\ (\mathrm{mod}\ #1)}

\newcommand{\E}[0]{\mathrm{E}}
\newcommand{\Var}[0]{\mathrm{Var}}
\newcommand{\Cov}[0]{\mathrm{Cov}}

\DeclareMathOperator{\di}{d\!}
\newcommand*\Eval[3]{\left.#1\right\rvert_{#2}^{#3}}

\renewcommand*{\bmod}{\mathbin{\%}}


%%%%%%%%%%%%%%%%%%%%%%%%%%%%%% Document Start %%%%%%%%%%%%%%%%%
\begin{document}
\section*{Sundry}

\begin{Answer}
    I typeset the homework using \LaTeX and discussed the problems with the following people.
    \begin{itemize}
        \item Yijia Chen -- yijia.chen@berkeley.edu
    \end{itemize}
\end{Answer}

\newpage
\Question{Markov’s Inequality and Chebyshev’s Inequality}

\begin{Parts}
    
    \Part $\E[X^2] = 13$
    \begin{Answer}
        $$\Var [X] = \E[X^2] - \E[X]^2 \implies \E[X^2] = \Var[X] + \E[x]^2 = 9 + 4 = 13$$
    \end{Answer}

    \Part $\Pr[X=2] > 0$
    \begin{Answer}
        This is not true. Say $X$ takes on values $a,b$ with probabilities $\frac{1}{2}$, respectively. 
        \begin{align*}
            \E[X] = 2   &\implies \frac{1}{2}a + \frac{1}{2}b = 2 \\
            \E[X^2] = 9 &\implies \frac{1}{2}a^2 + \frac{1}{2}b^2 = 13 
        \end{align*}
        We have $a = 5, b = -1$ as a solution, under which circumstances $\Pr[X=2] = 0$. 
    \end{Answer}

    \Part $\Pr[X \geq 2] = \Pr[X \leq 2]$
    \begin{Answer}
        This is not true. Say $X$ takes on values $a,b$ with probabilities $p, 1-p$, respectively. 
        \begin{align*}
            \E[X] = 2   &\implies pa + (1-p)b = 2 \\
            \E[X^2] = 9 &\implies pa^2 + (1-p)b^2 = 13
        \end{align*}
        We then know that $b = 2 \pm \frac{3}{\sqrt{\frac{1-p}{p}}}$. An example for this would be 
        $a = -7, b = 3, p = \frac{1}{10}$. 
    \end{Answer}

    \newpage
    \Part $\Pr[X \leq 1] \leq \frac{8}{9}$
    \begin{Answer}
        Since we know that $X$ is no larger than $10$, we define $Y = 10 - X$ so that $Y$ is a non-negative random variable. 
        \begin{align*}
            \Pr[X \leq 1] &= \Pr[Y \geq 9] \\
                          &\leq \frac{\E[Y]}{9} \\
                          &= \frac{8}{9}
        \end{align*}
    \end{Answer}

    \Part $\Pr[X \geq 6] \leq \frac{9}{16}$
    \begin{Answer}
        \begin{align*}
            \Pr[X \geq 6] &= \Pr[|X - \E[X]| \geq 4] - \Pr[X \leq -2] \\
                          &\leq \Pr[|X - \E[X]| \geq 4] \\
                          &\leq \frac{9}{16}
        \end{align*}
    \end{Answer}

    \Part $\Pr[X \geq 6] \leq \frac{9}{32}$
    \begin{Answer}
        Say $X$ only takes values $a,b$ with probabilities $p,1-p$. \\
        We have a counterexample as $\Pr[X=0] = \frac{9}{13}, \Pr[\frac{13}{2}] = \frac{4}{13}$
    \end{Answer}

\end{Parts}

\newpage
\Question{Subset Card Game}

\begin{Parts}

    \Part Prove that the probability that Yiming wins is at most $\frac{1}{8\alpha^2}$. 

    \begin{Answer}
        \begin{proof}
            Let the random variable $X$ denote the sum of cards in Yiming's hands. \\
            Let the random variables $I_i=1$ if card $i$ ends up in Yiming's hands and $I_i = 0$ otherwise.  
            \begin{align*}
                \Var[X] &= \Var[\sum_{i=1}^k I_ic_i] \\
                        &= \sum_{i=1}^k \Var[I_ic_i] \\
                        &= \sum_{i=1}^k c_i^2 \Var[I_i] \\
                        &= \frac{1}{4} \sum_{i=1}^k c_i^2 \\
                        &= \frac{1}{4} \\
                \Pr[X \geq \alpha] &= \frac{1}{2} \Pr[|X| \geq \alpha] \\
                                   &\leq \frac{1}{2} \frac{1}{4\alpha^2} \\ 
                                   &= \frac{1}{8\alpha^2}
            \end{align*}
        \end{proof}
    \end{Answer}

    \Part Find a deck of $k$ cards and target value $\alpha$ where the probability that Yiming wins is exactly $\frac{1}{8\alpha^2}$.

    \begin{Answer}
        Consider $k=3$. The deck is $\{\frac{\sqrt{2}}{2},0,-\frac{\sqrt{2}}{2}\}$. Let $\alpha=\frac{\sqrt{2}}{2}$. The winning probability is $\frac{1}{4}$. 
    \end{Answer}

\end{Parts}

\newpage
\Question{Sampling a Gaussian With Uniform}

\begin{Parts}
    
    \Part Prove that $-ln U_1 \sim Expo(1)$.

    \begin{Answer}
        \begin{proof}
            \begin{align*}
                \Pr[U_1 \leq a]      &= a \\
                \Pr[-\ln U_1 \leq a] &= \Pr[\ln U_1 \geq -a] \\
                                     &= \Pr[U_1 \geq e^{-a}] \\
                                     &= 1 - e^{-a}
            \end{align*}
            Thus we know that $-ln U_1 \sim Expo(1)$.
        \end{proof}
    \end{Answer}

    \Part Prove that $N_1^2 + N_2^2 \sim Expo(\frac{1}{2})$

    \begin{Answer}
        \begin{proof}
            \begin{align*}
                f(x,y) &= \frac{1}{2\pi} 2^{-\frac{1}{2}(x^2+y^2)} \\
                \Pr[N_1^2 + N_2^2 \leq r^2] &= \int \int f(x,y) dx dy \\
                                            &= \int_0^r \int_0^{2\pi} \frac{1}{2\pi} e^{-\frac{1}{2}r^2} d\theta dr \\
                                            &= 1 - e^{-\frac{1}{2}r^2}
            \end{align*}
            Thus we know that $N_1^2 + N_2^2 \sim Expo(\frac{1}{2})$. 
        \end{proof}
    \end{Answer}

    \Part How would you transform these two random variables into a normal random variable with mean 0 and variance 1?

    \begin{Answer}
        Let $-\frac{1}{2} \ln U_1$ represent $R^2$ since $R^2 \sim \exp(\frac{1}{2})$ \\
        Let $2\pi U_2$ represent the angle since angle is a uniformly distributed random variable $(0, 2\pi)$.
    \end{Answer}

\end{Parts}

\newpage
\Question{Optimal Gambling}

\begin{Parts}
    
    \Part What is $E[X_n]$ in terms of $X_0$, $p$, and $q$?
    \begin{Answer}
        \begin{align*}
            E[X_i] &= X_{i-1} + qX_{i-1}p - qX_{i-1}(1-p) \\ 
                   &= X_{i-1} + X_{i-1} q(2p-1) \\
                   &= (1 + 2pq - q) X_{i-1} \\
            E[X]   &= (1 + 2pq - q)^n X_0
        \end{align*}
    \end{Answer}

    \Part What value of $q$ will maximize $E[X_n]$?
    \begin{Answer}
        Since $2p - 1 > 0$, maximizing $q$ will maximize $E[X_n]$. We choose $q = 1$. 
        \begin{align*}
            \Pr[X_n = 2^n X_0] &= p^n \\
            \Pr[X_n = 0]       &= 1 - p^n
        \end{align*}
    \end{Answer}

    \Part Express $X_n$ in terms of $n,q,X_0,W_n$, where $W_n$ is the number of rounds you have won up until round $n$.
    \begin{Answer}
        If the player wins, the player's profit is $X_i = (1+q)X_{i-1}$. If the player loses, the profit is $X_i = (1-q)X_{i-1}$. 
        $$X_n = (1-q)^{n - W_n}(1+q)^{W_n}X_0$$
    \end{Answer}

    \Part Convergence? 
    \begin{Answer}
        It converges to the following expression. 
        \begin{align*}
            \lim_{n \to \infty} \ln X_n / n &= \lim_{n \to \infty} \frac{1}{n} (1-p)n \ln (1-q) + pn \ln (1+q) + \ln X_0 \\
                                            &= (1-p) \ln (1-q) + p \ln(1+q)
        \end{align*}
    \end{Answer}

    \newpage
    \Part Find the value of $q$ that maximizes your asymptotic growth rate. 
    \begin{Answer}
        \begin{align*}
            c   &= (1-p) \ln (1-q) + p \ln(1+q) \\
            e^c &= (1-q)^{1-p}(1+q)^p
        \end{align*}
        The above expression is maximized when $q = 2p - 1$. 
    \end{Answer}

    \Part Using the $q$ in the previous part and compute $E[X_n]$. 

    \begin{Answer}
        \begin{align*}
            E[X_n] = (2 - 2p)^n(1-p) 2p^{np}X_0 = 2^n(1-p)^{n - np}p^{np}X_0
        \end{align*}
    \end{Answer}

    \Part True or false
    \begin{Answer}
        It is not true. According to Chebyshev’s Inequality, we have 
        \begin{align*}
            n &\geq \frac{p(1-p)}{\epsilon^2\sigma} \\
              &= \frac{1}{4.01^2 \times 0.05} \\
              &= 500
        \end{align*}
        Sample space is too small. 
    \end{Answer}

    \Part True or false
    \begin{Answer}
        $p = 0.74$, $\frac{\sigma}{\sqrt{n}} = \frac{\sqrt{p(1-p)}}{n} \approx 0.04386$. \\
        The confidence interval is $[0.74 \pm 0.04386 \times 2] = [0.652, 0.828]$.
    \end{Answer}
\end{Parts}

\newpage
\Question{Boba in a Straw}

\begin{Parts}
    
    \Part What is the expected number of seconds that elapse before the straw is completely filled with boba for the first time?
    \begin{Answer}
        Using finite Markov chain to solve this problem. Let $t(x,y)$ denote the expected number of seconds for the straw to 
        be completely full if we start with top component having $x$ boba and the bottom component having $y$ boba. We then have the following equations. 
        \begin{align*}
            t(0,0) &= pt(0,1) + (1-p)t(0,0) + 1 \\
            t(0,1) &= pt(1,1) + (1-p)t(1,0) + 1 \\
            t(1,0) &= pt(0,1) + (1-p)t(0,0) + 1 \\
            t(1,1) &= 0
        \end{align*}
        We output $t(0,0)$ as the answer to this problem.
    \end{Answer}

    \Part Under these conditions, answer the previous part again. 
    \begin{Answer}
        Using finite Markov chain to solve this problem again. Let Let $t(x,y)$ denote the expected number of seconds for the straw to 
        be completely full if we start with top component having $x$ boba and the bottom component having $y$ boba. We then have the following equations. 
        \begin{align*}
            t(0,0) &= p(t(0,1) + 2) + (1-p)(t(0,0) + 1) \\
            t(0,1) &= p(t(1,1) + 3) + (1-p)(t(1,0) + 2) \\
            t(1,0) &= p(t(0,1) + 2) + (1-p)(t(0,0) + 1) \\
            t(1,1) &= 0
        \end{align*}
    \end{Answer}

    \Part What is the long-run average rate of Jonathan’s calorie consumption?
    \begin{Answer}
        In the long run the total number of boba consumed is just those coming into the bottom, so the rate is $10p$. 
    \end{Answer}

    \Part What is the long-run average number of boba which can be found inside the straw?
    \begin{Answer}
        The expected number of boba at the bottom is $p$, the expected number of boba at the bottom is $p$. Thus the 
        total number of boba inside the straw is $2p$. 
    \end{Answer}

\end{Parts}
    
\end{document}
