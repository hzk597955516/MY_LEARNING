%Template compiled by Alice lol
\documentclass[11pt]{article}
\usepackage{cs70}

%%%%%%%%%%%%%%%%%%%% name/id
\rfoot{\small Zehao Huang | 3033857597 | zehao@berkeley.edu}


%%%%%%%%%%%%%%%%%%%% Course/HW info
\newcommand*{\instr}{Babak Ayazifar and Satish Rao}
\newcommand*{\term}{Spring 2019}
\newcommand*{\coursenum}{CS 70}
\newcommand*{\coursename}{Discrete Mathematics and Probability Theory}
\newcommand*{\hwnum}{05}
\newcommand{\Mod}[1]{\ (\mathrm{mod}\ #1)}
\renewcommand*{\bmod}{\mathbin{\%}}


%%%%%%%%%%%%%%%%%%%%%%%%%%%%%% Document Start %%%%%%%%%%%%%%%%%
\begin{document}
\section*{Sundry}

\begin{Answer}
    I typeset the homework using \LaTeX and discussed the problems with the following people. 
    \begin{itemize}
        \item Yijia Chen -- yijia.chen@berkeley.edu
    \end{itemize}
\end{Answer}

\newpage
\Question{Squared RSA}

\begin{Parts}
    \Part Prove the identity $a^{p(p - 1)} \equiv 1 \pmod{p^2}$, where $a$ is coprime to $p$, and $p$ is prime.
    \begin{Answer}
        \begin{proof}
            We construct two sets $$S = \{1\cdot a,2 \cdot a,3\cdot a,\ldots,(p^2-1)\cdot a\} - \{p\cdot a,2p\cdot a,3p\cdot a,\ldots,(p-1)p\cdot a\}$$
            $$S' = \{1, 2, 3, \ldots, p^2 - 1\} - \{p, 2p, 3p, \ldots, (p-1)p\}$$ $\forall i \in S, \gcd(i, p^2)=1$. In addition, all elements in $S$ 
            have a different $\Mod{p^2}$ value. Suppose $ma > na \in S, ma = na \Mod{p^2}$, then $(m-n)a = 0 \Mod{p^2}$. According to our construction
            of $S$, $0 < m-n < p^2$. Thus $a \mid p^2$ must be true, which is a contradiction to the fact that $\gcd(a, p^2) = 1$. In addition, 
            $\forall i \in S$, $p \nmid i \implies p \nmid i \Mod{p^2}$. As a result, we can see that $S$ contains $S' \Mod{p^2}$. Let $f(S)$ be the 
            product of all elements in $S$, $f(S')$ be the product of all elements in $S'$, it must be true that $f(S) \equiv f(S') \Mod{p^2}$. 
            After canceling all the multipliers of $a$, we have $a^{p(p-1)} \equiv 1 \Mod{p^2}$. 
        \end{proof}

    \end{Answer}

    \Part Now consider the RSA scheme: the public key is $(N = p^2 q^2, e)$ for primes $p$ and $q$, 
          with $e$ relatively prime to $p(p-1)q(q-1)$. The private key is $d = e^{-1} \pmod{p(p-1)q(q-1)}$. 
          Prove that the scheme is correct for $x$ relatively prime to both $p$ and $q$, i.e.\ $x^{ed} \equiv x \pmod{N}$.
    \begin{Answer}
        \begin{proof}
            \begin{align*}
                d = e^{-1} \Mod{p(p-1)q(q-1)} &\implies de \equiv 1 \Mod{p(p-1)q(q-1)} \\
                                              &\implies \exists k \in \N, de = kp(p-1)q(q-1) + 1 \\
                                              &\implies x^{de} \equiv x^{kp(p-1)q(q-1) + 1} \Mod{N} \\ 
                                              &\implies x^{de} \equiv ((x^{p(p-1)})^{q(q-1)})^k \Mod{N}
            \end{align*}
            From part $(a)$ we know that $x^{p(p-1)} \equiv 1 \Mod{p^2}$. Thus $(x^{p(p-1)})^{q(q-1)} \equiv 1^{q(q-1)} \equiv 1 \Mod{p^2}$. 
            In addition, viewing $x^{p(p-1)}$ as a whole, $(x^{p(p-1)})^{q(q-1)} \equiv 1 \Mod{q^2}$. It is trivial to see that $\gcd(p^2, q^2) = 1$, 
            and as a result $(x^{p(p-1)})^{q(q-1)} \equiv 1 \Mod{p^2q^2}$. Thus $x^{de} \equiv 1^k \equiv 1 \Mod{N}$. 
        \end{proof}
    \end{Answer}

    \newpage
    \Part Prove that this scheme is at least as hard to break as normal RSA; that is, prove that if this scheme can be broken, 
          normal RSA can be as well. We consider RSA to be broken if knowing $pq$ allows you to deduce $(p - 1)(q - 1)$.
          We consider squared RSA to be broken if knowing $p^2q^2$ allows you to deduce $p(p - 1)q(q - 1)$.
    \begin{Answer}
        \begin{proof}
            We begin by assuming that $p^2q^2$ allows us to deduce $p(p-1)q(q-1)$. \\
            In the normal RSA scheme, if we know $pq$ in the first place, we can easily know $p^2q^2$ by computing $(pq)^2$. Then according to 
            the assumption, we can apply the algorithm to break squared RSA so that we know $p(p-1)q(q-1)$. Divide that by $pq$, we have $N$. 
            In addition to breaking the squared RSA problem, we just squared $pq$ and divided $p(p-1)q(q-1)$ by $pq$, which are easy tasks for
            computers. \\
            Thus we have shown that squared RSA is broken, normal RSA can be as well. Thus, this scheme is at least as hard to break as normal RSA. 
        \end{proof}
    \end{Answer}
\end{Parts}

\newpage
\Question{Breaking RSA}
Eve is not convinced she needs to factor $N = pq$ in order to break RSA. She argues: "All I need to know is $(p-1)(q-1)$... then I can find $d$
as the inverse of $e$ mod $(p-1)(q-1)$. This should be easier than factoring $N$." Prove Eve wrong, by showing that if she knows $(p-1)(q-1)$,
she can easily factor $N$ (thus showing finding $(p-1)(q-1)$ is at least as hard as factoring $N$). Assume Eve has a friend Wolfram, who can easily return the
roots of polynomials over $\R$ (this is, in fact, easy).

\begin{Answer}
    \begin{proof}
        We start by assuming that Eve already knows the value of $(p-1)(q-1)$. \\ 
        Because $(N,e)$ is the public key for RSA, Eve also knows the value of $pq$. Thus Eve knows the value of $p+q$ because 
        $p+q = pq + 1 - (p-1)(q-1)$. Let $M=p+q$. We construct the following polynomial. 
        $$P(x) = (x - p)(x - q) = x^2 - (p + q)x + pq = x^2 - Mx + N$$
        Since we know all the parameters of $P(x)$, the next thing Eve should do is hand the polynomial to Wolfram and let Wolfram
        calculate all $x$ where $P(x) = 0$. This gives us $p$ and $q$. \\ 
        Thus we have proved that if Eve knows $(p-1)(q-1)$, she can easily know $p$ and $q$, thus showing that finding $(p-1)(q-1)$ is at least
        as hard as factoring $N$. 
    \end{proof}
\end{Answer}

\newpage
\Question{Polynomial Practice}

\begin{Parts}
    \Part If $f$ and $g$ are non-zero real polynomials, how many roots do the following 
    polynomials have \textbf{at least}? How many can they have \textbf{at most}? 
    \begin{enumerate}[(i)]
        \item $f + g$
        \begin{Answer}
            Let the degree of $f$ be $d_1$ and the degree of $g$ be $d_2$. 
            \begin{itemize}
                \item At most: $\max(d_1, d_2)$
                \item At least: $\max(d_1, d_2) \Mod{2}$.
            \end{itemize}
        \end{Answer}

        \item $f\cdot g$
        \begin{Answer}
            Let the degree of $f$ be $d_1$ and the degree of $g$ be $d_2$. 
            \begin{itemize}
                \item At most: $d_1 + d_2$
                \item At least: $d_1 + d_2 \Mod{2}$.
            \end{itemize}
        \end{Answer}

        \item $f/g$, assuming that $f/g$ is a polynomial
        \begin{Answer}
            Let the degree of $f$ be $d_1$ and the degree of $g$ be $d_2$. 
            \begin{itemize}
                \item At most: $d_1 - d_2$
                \item At least: $d_1 - d_2 \Mod{2}$.
            \end{itemize}
        \end{Answer}
    \end{enumerate}

	\Part Now let $f$ and $g$ be polynomials over $\mathrm{GF}(p)$.
    \begin{enumerate}[(i)]
        \item We say a polynomial $f = 0$ if $\forall x, f(x) = 0$.
              If $f\cdot g = 0$, is it true that either $f=0$ or $g=0$?
        \begin{Answer}
            \textbf{Claim:} $f,g \in \mathrm{GF}(p), f \cdot g = 0 \implies f = 0 \lor g = 0$. 
            \begin{proof}
                The proof will be based on contraposition. \\
                Assume that neither $f = 0$ nor $g = 0$. Then $\exists x,y \in \{0,\ldots,p-1\}$, $f(x) \neq 0$, $g(y) \neq 0$.
                \begin{itemize}
                    \item $x=y$, then $f \cdot g(x) = f(x) \cdot g(x) \neq 0$. We have reached a contradiction. 
                    \item $x \neq y$, then $f \cdot g(xy) = f(xy)g(xy) \neq 0$. This is because $f(xy)$ can be written as the 
                    sum of the multiplication of a power of $y$ and all nonzero terms in $f(x)$. The same is true for $g$. Even if
                    one of $x$ and $y$ is zero, we can still add a multiply of $p$ to get a nonzero $x$ and $y$. We have reached a contradiction.  
                \end{itemize}
                Thus we conclude that $f,g \in \mathrm{GF}(p), f \cdot g = 0 \implies f = 0 \lor g = 0$.
            \end{proof}
        \end{Answer}

        \item If $\deg{f} \geq p$, show that there exists a polynomial $h$ with 
              $\deg{h} < p$ such that $f(x) = h(x)$ for all $x \in \{0,1,...,p-1\}$.
        \begin{Answer}
            \textbf{Claim:} $\forall x \in \{0,1,\ldots,p-1\}, \deg(f) \geq p, \exists h, \deg(h) < p, f(x) = h(x)$
            \begin{proof}
                Since $\deg{f} \geq p$, assume that $\deg{f} = n$. We construct $f$ as follows. 
                $$f(x) = a_0x^0 + a_1x^1 + x_2x^2 + \ldots + a_nx^n$$
                Since $p$ is prime, according Fermat's Little Theorem, $x^{p-1} = 1 \Mod{p}$. Thus for all the terms $a_ix^i$ where $i >= p$, 
                we can always find some integer $k, l$ such that $i = k(p - 1) + l$. Then we can rewrite the term $a_ix^i$ as 
                $a_ix^{k(p - 1) + l} = a_i(x^{p-1})^kx^l = a_ix^l \Mod{p}$. Since $l = x \Mod{p - 1}$, we know for sure that $l < p$. Thus we can
                just add $a_i$ to $a_l$, which does not change the polynomial. After processing all terms of $a_ix^i$ where $i \geq p$, we have 
                the new polynomial $P'$ with the maximum degree equal to $p - 1$. Thus we have proved the claim. 
            \end{proof}
        \end{Answer}
        
        \item How many $f$ of degree \textit{exactly} $d<p$ are there such that 
            $f(0) = a$ for some fixed $a\in\{0,1,\dots,p-1\}$?
        \begin{Answer}
            \textbf{Claim:} There are $(p - 1)p^{d-1}$ $f$ of degree exactly $d < p$ for $f(0) = a$. 
            \begin{proof}
                Consider the polynomial $f(x) = a_dx^d + \ldots + a_2x^2 + a_1x + a_0$. For each term $a_ix^i$ where $i \geq 1$ we can choose $a_i$ 
                to be any value from $\{0,1,2,\ldots,p\}$ except for $a_d$. To make the polynomial to have exactly degree $d$, $a_d \neq 0$. Thus 
                according to the multiplication principle, we have the total number of polynomials being $(p-1)p^{d-1}$. 
            \end{proof}
        \end{Answer}
    \end{enumerate}

    \Part Find a polynomial $f$ over $\mathrm{GF}(5)$ that satisfies 
    $f(0) = 1, f(2) = 2, f(4) = 0$. How many such polynomials are there?
    \begin{Answer}
        We construct the following delta polynomials.
        \begin{align*}
            \Delta_1 &= 2(x - 2)(x - 4) \Mod{5} \\
            \Delta_2 &= x(x - 4) \Mod{5} \\
            \Delta_3 &= 0 \Mod{5}
        \end{align*} 
        Then we can have an $f = \Delta_1 + 2\Delta_2 + 0\Delta_3 = 4x^2 - 20x + 16 = 4x^2 + 1 \Mod{5}$. \\
        We can have 25 such polynomials. We can choose $f(1),f(3)$ from $\{0,1,2,3,4\}$, and each combination gives us a unique polynomial. Thus
        the total number of possible polynomials is 25. 
    \end{Answer}
\end{Parts}

\newpage
\Question{Old secrets, new secrets}

\begin{Answer}
    We construct the polynomial $p$ from the $n+1$ points we know. 
    \begin{align*}
        p(x) &= p(1)\Delta_1(x) + p(2)\Delta_2(x) + p(3)\Delta_3(x) + \ldots + p(n + 1)\Delta_{n + 1}(x) \\
        p(x) &= a_0 + a_1x + a_2x^2 + a_3x^3 + \ldots + a_nx^n \\
        p(0) &= a_0 = s \\
        p(1) &= a_0 + a_1 + a_2 + a_3 + \ldots + a_n
    \end{align*}
    Support $p'(x)$ is the new polynomial that corresponds to $s'$. $p'(x)$ still passes the $n$ original points.
    \begin{align*}
        p'(x)        &= p'(1)\Delta_1(x) + p(2)\Delta_2(x) + p(3)\Delta_3(x) + \ldots + p(n + 1)\Delta_{n + 1} \\
        p'(x) - p(x) &= p'(1)\Delta_1(x) - p(1)\Delta_1(x) \\
        p'(0) - p(0) &= p'(1)\Delta_1(0) - p(1)\Delta_1(0) \\
        s' - s       &= (p'(1) - p(1)) \frac{(0 - 2)(0 - 3)(0 - 4)\ldots(0 - n - 1)}{(1 - 2)(1 - 3)(1 - 4)\ldots(1 - n - 1)} \\
        s' - s       &= (p'(1) - p(1))(n + 1) \\
        p'(1)        &= \frac{s' - s}{n + 1} + p(1)
    \end{align*}
    From the above calculation we know that Bob should report $\frac{s' - s}{n + 1} + p(1)$ to make the others 
    believe that the secret is $s'$. 
\end{Answer}

\end{document}
